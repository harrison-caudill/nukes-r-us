\section{Recommendation: Alternative Weapons}

To ensure that humanity can continue to kill one another, without
ending civilization, or denying access to space for generations,
``conventional'' arms are required.  Terrestrially, that means rifles
and expendable riflemen, that means tanks and TOW missiles.  In space,
that means \acp{asat} that do not generate debris; that means
SAFEKILLs.

The primary attribute being sought is the ability to disable or
otherwise deter, either temporarily or permanently, an enemy
spacecraft without generating debris.  One critical aspect of
temporary disablement/deterrence is the ability to do so for an
arbitrary length of time.  Ground-based jamming, for example, does not
qualify as it cannot be maintained when the satellite passes out of
view.  Space-based jamming from a nearby satellite, on the other hand,
does qualify as the attacker can shadow the target indefinitely.

Any attack method that, for example, caused the electrical systems of
an enemy spacecraft to falter would qualify.  Any attack method that
redirected the target into the atmosphere would also qualify.  A few
categories of potential weaponry are discussed.  All of the methods
listed below are possible using current levels of technology and
components available in commercial industry, and in some cases using
currently-deployed satellites.

While this discussion does most certainly revolve around the
weaponization of space, it is worth noting that nothing proposed here
appears to violate the Outer Space Treaty.\cite{outer-space}

\subsection{Blockade}
In response to Russia's invasion of Ukraine in 2022, the world levied
a vast array of economic sanctions against Russia and her
oligarchs.\cite{russian-economy} Economic sanctions, blockades, etc
are tools of statecraft that are far more nuanced than Tomahawk
missiles (precision-guided though they may be).  In a similar manner,
it seems worthy to carry the concept of a blockade over to the space
arena.  A blockade need not generate any debris; in fact, it need not
cause any damage at all.

Any spacecraft (commercial or military) which is independently capable
of close approach for an operation such as servicing or refueling
could, in principle, use those capabilities for \ac{rpo} with an
adversary.  Several spacecraft could fly formation with a target,
thereby limiting its maneuverability and interfering with its mission.
If a bloackading spacecraft is occluding a camera, then its \ac{isr}
mission cannot continue and a maneuver would be required to get a
clear view of the target area.  If accelerating in any direction would
result in a collision, however, then the target spacecraft would have
few options but to endure the blockade.  In addition, the interceptors
could shade the target's solar panels preventing them from charging
batteries or cover their thermal radiators making it more difficult
for them to cool themselves down.

In this manner, a strategic asset of an adversary could be effectively
blockaded.  Even if the target were to take evasive action in an
attempt to prevent a successful blockade, a small number of low-cost
and refuelable assets could continually harass the target, preventing
smooth mission operations, and potentially forcing it to expend all of
its valuable propellant.

While this approach is not expected to be useful against scores of
smallsats in a resilient constellation, it is likely to be effective
against large strategic assets. Since some laws of physics, such as
the lens radius necessary for \acf{hsi} will likely mandate large
strategic assets for some time, this approach is expected to remain
effective.

A spacecraft that is unable to execute its mission is just as useless
to an adversary as is one which has been destroyed with a \ac{kasat}.

\subsection{Physical Disablement}
Most terrestrial weaponry capable of physical disablement is kinetic
in nature.  That propensity seems to have carried forward to
counterspace weaponry: hit the satellite with a fast-moving chunk of
missile and the satellite explodes.  Physical disablement without
generating debris involves a great deal more finesse and control, and
usually the ability to operate in \ac{rpo}.  Several physical
characteristics of operations in orbit could be leveraged for the
production of these weapons.

\begin{itemize}

\item Thermal radiators are used to prevent overheating which may
  damage (among other things) the batteries, or may impede sensor
  efficiency (e.g. \acf{hsi}).  Depositing a material with a low
  emmissivity coefficient (such as gold) on the thermal radiators
  could significantly impare their efficiency and reduce the available
  thermal budget for operating systems.\footnote{Contrary to popular
  imagination, spacecraft frequently tend to be hot rather than cold.
  Without an atmosphere to accept excess heat, they're forced to shine
  out infrared energy into deep space.  Thermal regulation of
  spacecraft is a complicated topic.  For context, the AIAA's
  publication, Spcecraft Thermal Control Handbook Vol 1
  \cite{thermals} is over 800 pages long.}

\item A similar material could be deposited on the body of spacecraft
  itself.  By changing the Absorbtivity / Emissivity ratio, the amount
  of heat absorbed and retainedfrom the sun would be increased and raise the
  equilibrium temperature of the spacecraft, decreasing available
  thermal budget for operation of things like \ac{rf} power amplifiers
  for communications.

\item Exposed pins could be shorted.  For example, if \ac{cbus} pins
  were shorted together, then the effective resistance across the bus
  would drop to well below the nominal $60\Omega$ and the bus would
  likely cease to function.

\item Deposition of metal ions on antennas could disrupt antenna
  radiation patterns, potentially destroying \ac{rf} power amplifiers
  with reflected power and greatly reducing available \ac{rf} link
  margin thereby reducing available throughput of data.  If an
  \ac{isr} spacecraft is incapable of transmitting the received
  data, then it is incapable of fulfilling its mission.

\item Sensors could be occluded with deposited material, reducing or
  even destroying their efficacy.  If gold were deposited on a camera
  lens up to a depth of $\frac{\lambda}{2}$ then little or no light
  would reach the sensor.  Alternatively, a lesser amount could be
  deposited to only attenuate efficiency.
  
\item The same deposition technique could also occlude solar panels
  decreasing available electrical power budget for system operations.
  Terrestrial solar panels are frequently designed to be resilient
  against small occlusions (e.g. bird guano); it is doubtful that any
  spacecraft's solar panels are designed with a similar type of
  resilience.  There is potential for a small and focused occlusion on
  a target's solar panels to have an extremely outsized effect.

\item Grappling a target and forcibly de-orbiting it would, clearly,
  end its mission.  To ensure the target lacks available fuel for
  maneuvering, it could first be harassed with inexpensive satellites,
  forcing it to expend all of its propellant in evasive maneuvers
  before utilizing the high-cost grappling and de-orbit system.

\end{itemize}

As can be clearly seen, the ability to deposit a metal ion such as
gold on a spacecraft would provide a number of potential \acp{safe}.
\acp{kasat} are not necessary.


\subsection{Cyber Attacks}

Cyber attacks and jamming/spoofing are already a mainstay of military
action - as they should be.  In 2022, civilian space infrastructure
was attacked by Russia as part of their invasion of Ukraine.  A nation
at war has a significant incentive to conduct cyber attacks against
both civilian and military space infrastructure.\cite{big-risks} Also,
those same civilian organizations lack the necessary economic
incentives to secure their spacecraft as those cyber attacks are only
expected once it is too late - i.e. once a state of war exists.  Since
a specific cyber attack vector can usually be mitigated once known,
there is a strong incentive to avoid utilizing that attack until it
can be maximally effective.  The single-use nature of zero-day cyber
attacks could potentially make them less attractive than some of their
physical counterparts.  However, at the same time, the potential reach
and devastation of a ``Cyber Pearl Harbor'' style attack is also
worthy of consideration.

As was illustrated with the Russian hack of Viasat terminals, cyber
attacks are a real and effective means of disabling strategic
space-based capabilities without generating orbital debris.  A much
more thorough treatment of this topic can be found in
\href{https://www.researchgate.net/publication/341435541_Big_Risks_in_Small_Satellites_-_The_Need_for_Secure_Infrastructure_as_a_Service}{Big
  Risks in Small Satellites} which not only outlines the core issues
of economics but also outlines a market-friendly solution to that
problem.

Since many of the techniques described herein are already available to
deployed commercial assets, cyber attacks pose a very real risk to
national security, and a very real alternative to \acp{kasat}.
