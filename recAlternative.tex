\section{Recommendation: Alternative Weapons}

To ensure that humanity can continue to kill one another, without
ending civilization, or denying access to space for generations,
``conventional'' arms are required.  Terrestrially, that means rifles
and expendable riflemen, that means tanks and TOW missiles.  In space,
that means \acp{asat} that do not generate debris; that means
SAFEKILLs.

The primary attribute being sought is the ability to disable or
otherwise deter, either temporarily or permanently, an enemy
spacecraft without generating debris.  One critical aspect of
temporary disablement/deterrence is the ability to do so for an
arbitrary length of time.  Ground-based jamming, for example, does not
qualify as it cannot be maintained when the satellite passes out of
view.  Space-based jamming from a nearby satellite, on the other hand,
does qualify as the attacker can shadow the target indefinitely.

Any attack method that, for example, caused the electrical systems of
an enemy spacecraft to falter would qualify.  Any attack method that
redirected the target into the atmosphere would also qualify.  A few
categories of potential weaponry are discussed.  All of the methods
listed below are possible using current levels of technology in
commercial industry, and in some cases using currently-deployed
satellites.

While this discussion does most certainly revolve around the
weaponization of space, it is worth noting that nothing proposed here
appears to violate the Outer Space Treaty. (outer-space)

\subsection{Blockade}
In response to Russia's invasion of Ukraine in 2022, the world levied
a vast array of economic sanctions against Russia and her
oligarchs.(find some nytimes article or something summarizing
sanctions) Economic sanctions, blockades, etc are tools of statecraft
that are far more nuanced than Tomahawk missiles (precision-guided
though they may be).  In a similar manner, it seems worthy to carry
the concept of a blockade over to the space arena.  A blockade need
not generate any debris; in fact, it need not cause any damage at all.

Any spacecraft (commercial or military) which is independently capable
of close approach for an operation such as servicing or refueling
could, in principle, use those capabilities for rendezvous and
proximity operations (RPO) with an adversary.  Several spacecraft
could fly formation with a target, thereby limiting its
maneuverability and interfering with its mission.  If a bloackading
spacecraft is occluding a camera, then its Intelligence Surveillance
and Recoinnasance (ISR) mission cannot continue and a maneuver would
be required to get a clear view of the target area.  If accelerating
in any direction would result in a collision, however, then the target
spacecraft would have few options but to endure the blockade.  In
addition, the interceptors could interfere with the target's ability
to use its sensor payloads, or even shade its solar panels preventing
them from charging batteries or cover their thermal radiators making
it more difficult for them to cool themselves down.

While this approach is not expected to be useful against scores of
smallsats in a resilient constellation, it is likely to be effective
against large strategic assets. In this manner, a strategic asset of
an adversary could be effectively blockaded.  Even if the target were
to take evasive action in an attempt to prevent a successful blockade,
a small number of low-cost and refuelable assets could continually
harass the target, preventing smooth mission operations, and
potentially forcing it to expend all of its valuable propellant.

\subsection{Physical Disablement}
Most terrestrial weaponry capable of physical disablement is kinetic
in nature.  That propensity seems to have carried forward to
counterspace weaponry: hit the satellite with a fast-moving chunk of
missile and the satellite explodes.  Physical disablement without
generating debris involves a great deal more finesse and control, and
usually the ability to operate in close proximity to the target, known
as Rendezvous and Proximity Operations (RPO).

\subsection{Cyber/Jamming Attacks}
Probably going to rewrite this section.

Cyber attacks and jamming/spoofing are already a mainstay of military
action - as they should be.  In 2022, civilian space infrastructure
was attacked by Russia as part of their invasion of Ukraine.  As was
outlined in the 2019 publication Big Risks in Small Satellites, a
nation at war has a significant incentive to conduct cyber attacks
against both civilian and military space infrastructure.  Also, those
same civilian organizations lack the necessary economic incentives to
secure their spacecraft as those cyber attacks are only expected once
it is too late - i.e. once a state of war exists.  Since a specific
cyber attack vector can usually be mitigated once known, there is a
strong incentive to avoid utilizing that attack until it can be
maximally effective.  The single-use nature of cyber attacks could
potentially make them less attractive than some of their physical
counterparts.

As was illustrated with the Russian hack of Viasat terminals, cyber
attacks are a real and effective means of disabling strategic
space-based capabilities without generating orbital debris.  Also as
discussed in Big Risks in Small Satellites, spacecraft could be
hijacked via cyber attack.  It is possible, for example, to hijack
commercial spacecraft with the existing ability to represent a
national security risk.  The subsequent publication Commercial Space
System Security Guidelines begins the discussion of appropriate cyber
defenses for commercial spacecraft.
