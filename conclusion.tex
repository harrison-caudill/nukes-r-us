\section{Conclusion \& Further Research}

\ac{kin} is arguably the oldest, most fundamental, and most effective
approach to weapons development; it is also the most problematic in
the orbital warfighting domain.  The world is unlikely to ever succeed
at banning any useful \acp{kasat} to say nothing of banning the entire
approach.  Furthermore, given existing geopolitical tensions between
the various nations with \ac{kasat} capabilities, it is unlikely that
there would even be cooperation on this topic.  Assuming it is not
possible to ban \acp{kasat}, the next best approach is to mitigate
their harm in the same way that the harm from nuclear weapons is
mitigated: transition them to the status of ``strategic deterrent''.

The recommended mitigations are all independently effective, can all
be executed unilaterally, and work in combination to greater effect
than the sum of the parts.


\subsection{Actionable Recommendations}

\subsubsection{Internal Governance}

\textbf{Escalation Policy:} The circumstances under which it is
considered reasonable to launch \acp{kasat} should be articulated, and
enforced.

\textbf{Chain of Command:} The set of officers and leaders with
authority to order the launch of \acp{kasat} should be clearly
defined, and those officers/leaders should be well-trained in the
consequences of their actions.

\textbf{Target Selection:} It appears as though China tested one of
their \acpos{dasat} for use against an American \ac{gps} satellite,
such an attack would generate debris that is likely to live for
millenia and endanger some of the most valuable capabilities that all
citizens from all nations currently enjoy.  It might be reasonable,
subject to an escalation policy, to avoid targets at higher orbital
altitudes.

\textbf{Attack Methodology:} If, as seems likely, multiple attack
vectors are available, the least destructive vector should be chosen.
For example, a head-on collision will generate a very different debris
profile than a strike in the nadir (facing the center of the Earth)
direction.

\textbf{De-Escalation:} Similar to nuclear de-escalation channels, a
method of communicating with command personnel at nations equipped
with \acp{kasat} should be established.

\textbf{No-First-Strike:} Similar to nuclear armaments, nations should
pledge to not be the first to launch a \ac{kasat}.

\subsubsection{External Governance}

\textbf{Testing Limitations:} Agree upon a quota structure which not
only limits the potential impact of any nation's testing, but also
incentivizes them to develop less-destructive approaches.

\textbf{Response Policies:} Ensure that other nations understand the
military and economic consequences of using \acp{kasat} in anger, or
exceeding any testing limitations.

\subsubsection{Alternative Weapons Development}
The mitigation strategy likely to be most controversial is the concept
of helping to ensure that adversaries are in possession of more
weaponry.  It may be true that the world is a better place if nations
like Russia and China, who have a record of irresponsible behavior in
space, were in possession of weapons utilizing a \ac{safe} approach,
but the question of how or whether to pursue that end is a delicate
question.  At the very least, nations can pursue internal weapons
development programs in the hopes that adversaries will do the same --
it seems likely that adversaries would wish to develop weapons to
match.


\subsection{Further Research}
\label{section::furtherWork}

\begin{itemize}

\item A full analysis of the command-and-control pathways for asat
  systems in the various nations, with the cooperation of experts in
  the field and/or governments.

\item A game-theory analysis of \acp{kasat} in the context of
  strategic deterrence.

\item An exploration of the role of ``civilian oversight'' in
  management of \acp{kasat}.

\item A more sophisticated analysis of sustainable testing policies.

\item Analysis of available \ac{ssa} systems in the context of using
  them as part of a monitoring and enforcement system.

\item An exploration of politically viable quota systems.

\end{itemize}


\subsection{Conclusion}

\ac{kin} has been the premier kill mechanism since the dawn of human
warfare and isn't going out of style anytime soon.  The course of
\ac{asat} weapons development has followed the familiar pattern of
finding ways to apply kinetic energy to the target.  A great deal of
analysis and diplomatic effort has been focused upon the currently
most popular form of \ac{kasat}: the \acf{dasat}.  The general message
of this international conversation seems largely focused upon the
concept of eliminating them.  This approach has two problems that
disqualify it from being fruitful: the objective is simultaneously
inadequate and unacheivable.

\acp{kasat} (like the \ac{dasat}) are useful military weapons.  Useful
weapons, as a rule, do not get banned, especially if no alternatives
exist.  Even cluster munitions (for which there are numerous effective
alternatives) have recently come back in style with their use on
civilians in Ukraine.  Banning \acp{dasat} is likely to remain
unacheivable until they are no longer useful.  Even if it were
possible right away, it would be woefully inadequate.  The threat of
runaway debris growth is from the general category of \ac{kin}
weapons, which includes a whole host of weapons outside of the
\ac{dasat}.  If even a small subset of \ac{kasat} cannot be
effectively banned, then efforts should instead be turned to
mitigation and management.

The incentives to develop and field \acp{kasat} are strong because
they work, because there isn't much else to use in their place, and
because there is very little in the way of effective pushback.  The
independent, unilateral, and synergistic recommendations above
directly address the incentive structure of \acp{kasat}.  It may feel
right to try to ban them, but humanity's progress has not come through
ideological purity so much as through pragmatic reality.  We aren't
going to succeed at banning \aclp{kasat}, so let's at least ensure
that they stay on the shelf.
