\section{Conclusion \& Further Research}

\ac{kin} is arguably the oldest, most fundamental, and most effective
approach to weapons development; it is also the most problematic in
the orbital warfighting domain.  The world is unlikely to ever succeed
at banning even one type of \ac{kasat} to say nothing of banning the
entire approach.  Furthermore, given existing geopolitical tensions
between the various nations with \ac{kasat} capabilities, it is
unlikely that there would even be cooperation on this topic.  Assuming
it is not possible to ban \acp{kasat}, the next best approach is to
mitigate their harm in the same way that the harm from nuclear weapons
is mitigated: transition them to the status of ``strategic
deterrent''.

The recommended mitigations are all independently effective, can all
be executed unilaterally, and work in combination to greater effect
than the sum of the parts.


\subsection{Actionable Recommendations}

\subsubsection{Internal Governance}

\textbf{Escalation Policy:} The circumstances under which it is
considered reasonable to launch \acp{kasat} should be articulated, and
enforced.

\textbf{Chain of Command:} The set of officers and leaders with
authority to order the launch of \acp{kasat} should be clearly
defined, and those officers/leaders should be well-trained in the
consequences of their actions.

\textbf{Target Selection:} It appears as though China tested one of
their \acpos{dasat} for use against an American \ac{gps} satellite,
such an attack would generate debris that is likely to live for
millenia and endanger some of the most valuable capabilities that all
citizens from all nations currently enjoy.  It might be reasonable,
subject to an escalation policy, to avoid targets at higher orbital
altitudes.

\textbf{Attack Methodology:} If, as seems likely, multiple attack
vectors are available, the least destructive vector should be chosen.
For example, a head-on collision will generate a very different debris
profile than a strike in the nadir (facing the center of the Earth)
direction.

\textbf{De-Escalation:} Similar to nuclear de-escalation channels, a
method of communicating with command personnel at nations equipped
with \acp{kasat} should be established.

\textbf{No-First-Strike:} Similar to nuclear armaments, nations should
pledge to not be the first to launch a \ac{kasat}.

\subsubsection{External Governance}

\textbf{Testing Limitations:} Agree upon a quota structure which not
only limits the potential impact of any nation's testing, but also
incentivizes them to develop less-destructive approaches.

\textbf{Response Policies:} Ensure that other nations understand the
military and economic consequences of using \acp{kasat} in anger, or
exceeding any testing limitations.

\subsubsection{Alternative Weapons Development}
The mitigation strategy likely to be most controversial is the concept
of helping to ensure that adversaries are in possession of more
weaponry.  It may be true that the world is a better place if places
like Russia and China, who have a record of irresponsible behavior in
space, were in possession of weapons utilizing a \ac{safe} approach,
but the question of how or whether to pursue that end is a delicate
question.  At the very least, nations can pursue internal weapons
development programs in the hopes that adversaries will do the same --
it seems unlikely that an adversary will \emph{not} wish to develop
weapons to match.


\subsection{Further Research}
\label{section::furtherWork}

\begin{itemize}

\item A full analysis of the command-and-control pathways for asat
  systems in the various nations, with the cooperation of experts in
  the field and/or governments.

\item A game-theory analysis of \acp{kasat} in the context of
  strategic deterrence.

\item An exploration of the role of ``civilian oversight'' in
  management of \acp{kasat}.

\item A more sophisticated analysis of sustainable testing policies.

\item Analysis of available \ac{ssa} systems in the context of using
  them as part of a monitoring and enforcement system.

\item An exploration of politically viable quota systems.

\end{itemize}


\subsection{Conclusion}

It is possible, if not likely, that the \ac{dasat} in its present form
will soon be obsolete; perhaps hypersonic missiles become the next
\ac{kasat} of choice, or perhaps SpinLaunch will be modified to launch
ballistic payloads for ship-to-ship and ship-to-satellite war, or
perhaps \acp{coasat} will take over.  Whatever the future may hold for
\acp{dasat}, the cat-and-mouse game of \ac{kin} weapons development is
unlikely to end anytime soon.  First came the bullet, then armor, and
then the armor-piercing round.  If that pattern continues, then the
same technologies that make launch cheaper would likely be used to
produce cheaper \acp{dasat} or to launch more \acp{coasat}.  As every
schoolchild knows, sticks and stones will break bones and that same
logic applies to satellites.  As long as \acp{kasat} are effective,
they're a risk to be managed.  However, unlike the terrestrial
warfighting domains, the consequences of \ac{kin} attack can linger
for a very long time.

A ban on \aclp{dasat} would be analogous to a ban on fission bombs
delivered by bombers; it wouldn't do much about \acp{icbm} for example
or hydrogen bombs.  It isn't enough to ban just one type of \ac{asat}
missile when the entire approach of kinetic attack needs to be banned.
Much like nuclear weapons, that ban will almost certainly never
happen.  Nuclear weapons are kept at bay largely by combinations of
internal and external governance, the consequences of \ac{mad}, and
the availability of alternative (i.e. conventional) weaponry; they are
not kept at bay by any ban.  A ban on the use of \ac{kin} seems
exceedingly unlikely, and so an alternative is proposed: don't try to
ban \acp{kasat}, try to manage them.

The incentives to develop and field \acp{kasat} are strong because
they work, because there isn't much else to use in their place, and
because there is very little in the way of effective pushback.  A
cyber attack may or may not work, and if it does work it may or may
not be obvious to the attacker that it did, and it is quite likely
that the target would quickly adapt and patch the vulnerability.
LASER dazzling usually doesn't work when the target is out of range,
and if it is capable of damaging an optical sensor the rest of the
satellite would still function.  For a complete and permanent victory,
the world currently has only \acp{kasat}.  If the objective is to
decrease the risk to the world of \acp{kasat} then the incentives will
need to be changed.
