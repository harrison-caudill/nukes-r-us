\section{Conclusion \& Further Research}

Humans are, as a general rule, incentivized to act by way of personal
incentives.  While seemingly tautological in nature, that maxim holds
great power.  Instead of attempting to cajole world leaders to ban
\acp{kasat} with statements of the form ``It's the right thing to do''
it is recommended that incentive structures be modified to make them
unattractive in practice.  Much like nuclear weapons, \acp{kasat}
should be fully relegated to the shelf as a strategic deterrent.

\subsection{Actionable Recommendations}

Two aspects of the incentive structure are highly remediable:
availability of alternative weapons and internal/external governance.

Any meaningful ban of \acp{kasat} is extremely unlikely to happen.
The incentives to develop/deploy them is simply too strong.  Instead
of trying to ban these weapons, bring oversight and management to
them.  Internally, that can mean published military doctrine with
elements such as:

\textbf{Escalation Policy:} The circumstances under which it is
considered reasonable to launch ASAT missiles should be articulated,
and enforced.

\textbf{Chain of Command:} The set of officers and leaders with
authority to order the launch of ASAT missiles should be clearly
defined, and those officers/leaders should be well-trained in the
consequences of their actions.

\textbf{Target Selection:} While it appears as though China tested one
of their \acpos{dasat} for use against an American \ac{gps} satellite,
such an attack would generate debris that is likely to live for
millenia and endanger some of the most valuable capabilities that all
citizens from all nations currently enjoy.  It might be reasonable,
subject to an escalation policy, to avoid targets at higher orbital
altitudes.

\textbf{Attack Methodology:} If, as seems likely, multiple attack
vectors are available, the least destructive vector should be chosen.
For example, a head-on collision will generate a very different debris
profile than a strike in the nadir (facing the center of the Earth)
direction.

\textbf{De-Escalation:} Similar to nuclear de-escalation channels, a
method of communicating with command personnel at nations equipped
with \acp{kasat} should be established.

\textbf{No-First-Strike:} Similar to nuclear armaments, nations should
pledge to not be the first to launch a \ac{kasat}.

\noindent External pressure can also be applied:

\textbf{Testing Limitations:} Agree upon a quota structure which not
only limits the potential impact of any nation's testing, but also
incentivizes them to develop less-destructive approaches.

\textbf{Response Policies:} Ensure that other nations understand the
military and economic consequences of using \acp{kasat} in anger, or
exceeding any testing limitations.

In addition to the steps above to modify governance structures,
nations should aggressively pursue the development of \acp{sasat}.


\subsection{Further Research}
\label{section::furtherWork}

\begin{itemize}

\item A full analysis of the command-and-control pathways for asat
  systems in the various nations, with the cooperation of experts in
  the field and/or governments.

\item A game-theory analysis of \acp{kasat} in the context of
  strategic deterrence.

\item An exploration of the role of ``civilian oversight'' in
  management of \acp{kasat}.

\item A more sophisticated analysis of sustainable testing policies.

\item Analysis of available \ac{ssa} systems in the context of using
  them as part of a monitoring and enforcement system.

\item An exploration of politically viable quota systems.

\end{itemize}


\subsection{Conclusion}

It is possible, if not likely, that the \ac{dasat} in its present form
will soon be obsolete; perhaps hypersonic missiles become the next
\ac{kasat} of choice, or perhaps SpinLaunch will be modified to become
ballistic payloads for ship-to-ship and ship-to-satellite war, or
perhaps \acp{coasat} will take over.  Whatever the future may hold for
\acp{dasat}, the cat-and-mouse game of \ac{kin} weapons development is
unlikely to end anytime soon.  First came the bullet, then armor, and
then the armor-piercing round.  If that pattern continues, then the
same technologies that make launch cheaper would likely be used to
produce cheaper \acp{dasat} or to launch more \acp{coasat}.  As every
schoolchild knows, sticks and stones will break bones; the same logic
applies to satellites.  As long as kinetic weapons are effective,
they're a risk to be managed.

A ban on \aclp{dasat} would be analogous to a ban on fission bombs
delivered by bombers; it wouldn't do much about \acp{icbm} or hydrogen
bombs.  As history has shown, however, banning weapons is difficult at
best.  Banning an effective weapon such as \acp{dasat} is hard enough,
but what is truly sought is a ban on the entire approach of \ac{kin}.
Nuclear weapons are kept at bay largely by combinations of internal
and external governance, the consequences of \ac{mad}, and the
availability of alternative (i.e. conventional) weaponry; they are not
kept at bay by any ban.  A ban on the use of \ac{kin} seems
exceedingly unlikely, and so an alternative is proposed: don't try to
ban \acp{kasat}, try to manage them.

The incentives to develop and field \acp{kasat} is strong because they
work, because there isn't much else to use in their place, and because
there is very little in the way of effective pushback.  A cyber attack
may or may not work, and if it does work it may or may not be obvious
to the attacker that it did.  LASER dazzling usually doesn't work when
the target is out of range, and if it is capable of damaging an
optical sensor the rest of the satellite would still function.  For a
complete and permanent victory, the world currently has only
\acp{kasat}.  If the objective is to decrease the risk to the world of
\acp{kasat} then the incentives will need to be changed.
