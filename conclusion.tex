\section{Conclusion \& Further Research}

While it is possible, if not likely, that the \ac{dasat} as we know it
now will soon be obsolete as the world transitions away from singular
assets and towards resilient constellations.  While there will likely
always be some assets whose operational parameters are bound by the
laws of physics making them large and expensive, it may become too
unweildy or too expensive to build/maintain rockets designed decades
ago.  What is unlikely to change, however, is the nature of war.
First comes the bullet, then the bulletproof vest, and then the
armor-piercing round.  The cat-and-mouse game of weapons development
has a long tradition in human history, and there are few reasons to
believe that it will end now.  If that pattern continues, then the
same technologies that make launch cheaper would likely be used to
produce cheaper \acp{dasat} or to launch more \acp{coasat}.  As every
schoolchild knows, sticks and stones will break bones; the same logic
applies to satellites.  As long as kinetic weapons are effective,
they're a risk to be managed.

A ban on \aclp{dasat} would be analogous to a ban on fission bombs
delivered by bombers; it wouldn't do much about \acp{icbm} or hydrogen
bombs.  As history has shown, however, banning weapons is difficult at
best.  Banning an effective weapon such as \acp{dasat} is hard enough,
but what is truly sought is a ban on the entire approach of \ac{kin}.
Nuclear weapons are kept at bay largely by combinations of internal
and external governance, the consequences of \ac{mad}, and the
availability of alternative (conventional) weaponry; they are not kept
at bay by any ban.  A ban on the use of \ac{kin} seems exceedingly
unlikely, and so an alternative is proposed: don't try to ban
\acp{kasat}, try to manage them.

The incentives to develop and field \acp{kasat} is strong because they
work, and because there isn't much else that does.  A cyber attack may
or may not work, and if it does work it may or may not be obvious to
the attacker that it did.  LASER dazzling usually doesn't work when
the target is out of range, and if it is capable of damaging an
optical sensor the rest of the satellite would still function.  For a
complete and permanent victory, the world currently has only
\acp{kasat}.  If the objective is to decrease the risk to the world of
\acp{kasat} then the incentives will need to be changed.


\subsection{Recommendations}

Two aspects of the incentive structure are highly remediable:
availability of alternative weapons and internal/external governance.

Any meaningful ban of \acp{kasat} is extremely unlikely to happen.
The incentives to develop/deploy them is simply too strong.  Instead
of trying to ban these weapons, bring oversight and management to
them.  Internally, that can mean published military doctrine with
elements such as:

\textbf{Escalation Policy:} The circumstances under which it is
considered reasonable to launch ASAT missiles should be articulated,
and enforced.

\textbf{Chain of Command:} The set of officers and leaders with
authority to order the launch of ASAT missiles should be clearly
defined, and those officers/leaders should be well-trained in the
consequences of their actions.

\textbf{Target Selection:} While it appears as though China tested one
of their \acpos{dasat} for use against an American \ac{gps} satellite,
such an attack would generate debris that is likely to live for
millenia and endanger some of the most valuable capabilities that all
citizens from all nations currently enjoy.  It might be reasonable,
subject to an escalation policy, to avoid targets at higher orbital
altitudes.

\textbf{Attack Methodology:} If, as seems likely, multiple attack
vectors are available, the least destructive vector should be chosen.
For example, a head-on collision will generate a very different debris
profile than a strike in the nadir (facing the center of the Earth)
direction.

\textbf{De-Escalation:} Similar to nuclear de-escalation channels, a
method of communicating with command personnel at various ASAT-armed
adversaries should be created.

\textbf{No-First-Strike:} Similar to nuclear armaments, nations should
pledge to not be the first to launch a \ac{kasat}.

\noindent External pressure can also be applied:

\textbf{Response Policies:} Ensure that other nations understand the
consequences of using \acp{kasat} in anger.

\textbf{Testing Limitations:} Agree upon a quota structure which not
only limits the potential impact of any nation's testing, but also
incentivizes them to develop less-destructive approaches.  Violations
could be met with escalating pressure.

Combined with the development of alternative weapons, these aspects of
governance can provide the necessary disincentives to use as well as
the necessary safeguards to prevent unfortunate accidents.

\subsection{Further Research}

\begin{itemize}

\item A full analysis of the command-and-control pathways for asat
  systems in the various nations

\item A game-theory analysis of ASAT missiles and what ``civilian
  oversight'' even means WRT launch risk (talk to BBdM, though you'll
  have to pay him)

\item A more sophisticated analysis of sustainable testing policies

\item it might be useful here to outline the game theoretic effects of
  public doctrine (positively stabilizing) as well as destabilizing
  effects from the RAND literature. it's a complex subject but you
  need to back up your assertion that doctrine should be published
  with an analysis of why it would have a stabilizing effect.

\end{itemize}
