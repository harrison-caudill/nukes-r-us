\titlenote{EXECUTIVE SUMMARY}

\maketitle

%% These are fancyheader styles defined in the preamble.  We set the
%% general style, then the style for the first page.  We do the same
%% again for the paper itself.
\pagestyle{theRest}
\thispagestyle{firstPage}

%%
%% Yeah...I know...I should be using relative locations
\textblockorigin{1in}{\paperheight-.85in}

\begin{textblock*}{2.5in}(0in, 0in)
\tiny
{\noindent\copyright 2022 Harrison Caudill
All Rights Reserved
}
\end{textblock*}

\begin{textblock*}{2.5in}(5.5in, 0in)
\tiny
    {\noindent
      \hfill
      Revision:
      \input{rev}
      \hspace{1in}
    }
\end{textblock*}


\addcontentsline{toc}{section}{Executive Summary}

The general consensus among space industry professionals and
scientists is that \ac{asat} missiles should be banned.  The resulting
debris from a successful strikes poses significant safety, economic,
and operational risks to orbial assets.  Unlike terrestrial
warfighting domains where debris simply falls to the ground where it
can either be cleaned up or ignored, debris in orbit can remain in
orbit for years, if not millenia.  On numerous occasions, various
organizations around the world have even called for such a ban.
However, not only do these efforts fall short of the necessary scope
to be effective, banning debris-generating weapons in their current
forms but not the practice of turning target spacecraft into puffs of
debris, but they are also exceedingly unlikely even to accomplish
their stated goals.  \acp{kasat}, whose purpose is to transform
targets into clouds of debris, will likely never be banned for the
same reasons as nuclear weapons; the current weapons, and more
importantly the approach, both work.  If neither the approach, nor the
weapons can be banned, then efforts should be turned to the problem of
mitigating their destructive potential.  This paper examines the
incentive structures associated with \acp{kasat} and finds concrete
steps which can be taken to alter those incentives, and reduce the
risk of catastrophic deployment of \acp{kasat}.

While nuclear weapons may exist in abundance, they largely go unused
in their silos.  While far from perfect, the arrangement of strategic
deterrence, or \ac{mad}, is at least effective at preventing a
large-scale nuclear war.  A similar equilibrium may be sought for
\acp{kasat}.  It might not be possible to eliminate them entirely, but
if they can be relegated to the station of strategic deterrent, then
they too might go unused.  At present, however, the incentives to
deploy them are strong, the governance and control of them weak, the
alternatives practically non-existent, and the consequences of use
are, at best, unknown.  The actionable recommendations below directly
address these aspects of the incentive structure.

\section*{\ac{asat} Missiles and \acf{kasat}}

The fundamental concept of an \ac{asat} missile (and indeed with most
debris-generating weapons) is relatively straight-forward, and follows
the general trend of weapons-development throughout history: apply
more kinetic energy than the target can safely asorb or deflect.
\ac{asat} missiles hit the target with a \ac{kkv}, grenades hit the
target with shrapnel, guns with bullets, etc.  The primary reason the
world wishes to ban \ac{asat} missiles is because the resulting debris
doesn't simply settle to the ground where it can be swept up and
discarded; it remains in orbit and threatens other spacecraft.  Humans
in orbit already require a special type of shielding to protect them
from collision with space debris, and the \ac{iss} has already
replaced several windows due to impacts.  The relative speed of a
head-on collision in \ac{leo} is approximately 36,000mph.  At that
speed, even a fleck of paint is enough to irreparably damage a
spacecraft window.

\ac{asat} missiles come in two flavors; the \acf{coasat} and the far
more common \acf{dasat}.  \acp{coasat} are already in orbit allowing
them to attack at any moment, but requiring usually half an orbital
period to comlete.  \acp{dasat} are launched from the ground allowing
for rapid interception, but with the caveat that the target must be
nearby.  Both types of missiles have one central purpose: to
physically collide with the target at many thousands of miles per hour
turning both objects into puffs of debris like so much shrapnal.
\ac{asat} missiles may be the most common weapons of this type, but
they are by no means alone.  The Soviet Union launched a satellite
with a machine gun, and also detonated spacecraft near their targets
like orbital mortar rounds.

While the world's attention may be on \ac{asat} missiles, it would be
more appropriate to consider the entire category of kinetic weapons.
Any weapons which, through application of kinetic energy, \emph{do}
generate debris in orbit will be referred to as a \acp{kasat}.
Weapons which \emph{do not} generate debris will be referred to as
\acp{sasat}.  This distinction is alarmingly poorly managed.  For
example, the recent motion at the UN to ban \acp{dasat} does not even
attempt to ban the other common type of \ac{asat} missile, the
\ac{coasat}, let alone ban all debris-generating weapons.
Incidentally, Russia, China, and North Korea all voted against that
measure.  While, the \ac{dasat} in its present form may have seen the
peak of its utility as the world transitions most orbital assets away
from large single targets and towards resilient constellations, the
fundamental principle of applying kinetic energy has not lost utility,
nor is it likely to.

\section*{Kessler Syndrome}

Much like an uncontrolled fission reaction in a nuclear bomb, the
potential for exponential growth of debris rests tenuously upon the
current status, distribution, and management of what is already in
orbit.  That exponential ``runaway'' increase of debris, known as
``Kessler Syndrome'', is what the world wishes to avoid.  Sadly, it is
already too late to ``avoid'', as it has already started; \ac{asat}
missiles simply make the problem more acute more quickly.  Kessler et
al predict that the total number of collisions, over time, will grow
exponentially once enough total mass is in orbit.  That critical point
was reached over a decade ago, meaning that the exponential increase
has already begun.  It may not ``seem'' like there is an exponential
increase in collision rates because the time between collisions is
still measured in years.  However, absent active mitigations, that
number will steadily shrink.


\section*{Incentives}

If \ac{asat} missiles are of utility to militaries then attempts to
ban them are likely to meet with dismal results and are a waste of
time.  If \ac{asat} missiles are not of utility, then attempts to ban
them are unnecessary, and (mostly) a waste of time.  Either way,
attempting to ban \ac{asat} missiles is a waste of time and
opportunity.  Attempts to attenuate the odds of their use, however,
may yet succeed.

\subsection*{Efficacy: They Get the Job Done}

Application of kinetic energy is a weapons principle that stretches
back to the very first hominid ancestor using a stick.  It is a
principle so firmly rooted in physical laws that it is all but
guaranteed to be an effective tactic stretching as far into the future
as even science fiction writers dare to dream.  Application of kinetic
energy, is a weapons approach that works, and always will.

The most common type of \ac{kasat}, at present, is the \ac{dasat}.
While the cost effectiveness of \acp{dasat} is waning as the space
industry transitions from large and expensive satellites to resilient
constellations of small and inexpensive satellites, the approach is
still fundamentally sound.  Current \acp{dasat} may be expensive and
esoteric, but the same scientific and industrial breakthroughs that
bring cheaper launches can also be applied to weapons development.
Several physical laws, however, suggest that there will continue to be
some number of large and expensive satellites.  \acp{hsi}, for
example, will likely always require a large lens as the number of
available photons is relatively small.


\subsection*{Availability: Few Viable Alternatives Exist}

A distinction should be drawn between a \acf{kasat} a \acf{sasat}.  A
\ac{kasat} is a debris-generating weapon, while a \ac{sasat} is not.
Extensive use of \acp{kasat} would generate large quantities of
problematic debris in orbit, whereas widespread use of \acp{safe}
would not.

Two strategic attributes exist for \acp{kasat}: their effects are
permanent and their destructive power is complete.  Jamming
communications from ground-terminals, for example, is not permanent as
it requires that the target spacecraft be within range of the
transmitter.  Similarly, using LASERs to physically damage an optical
sensor is not complete as the non-optical systems would remain
functional.

If a viable alternative were to be sought for \acp{kasat}, then it
would likely require those attributes: permanence and completeness.
If the mission is to completely, and permanently take a strategic
orbital asset off the board in time of war: \acp{kasat} are the only
widely-available and easily-understood method.


\subsection*{Governance: Few Internal or External Controls}

\ac{cnc} safeguards against accidental deployment of nuclear weapons
are well publicized, well studied, and at least somewhat understood.
By contrast, no \ac{cnc} safeguards against inadvertent launch of a
\ac{kasat} are known.  While the world may generally condemn the
deployment of \acp{kasat}, how much weight does that condemnation
really carry?  Russia, for example, was widely condemned for their
2021 \ac{kasat} test and also for their 2022 invasion of Ukraine.  It
seems likely that they would have known in advance that such
condemnation would occur, and they took those steps anyway.  Unlike
nuclear weapons, response and escalation policies do not seem to
exist and certaily aren't enforced.


\section*{Recommended Mitigations}

Strong words and fervent hopes are as likely to affect the outcome of
deployment of \acp{kasat} as they have at altering the course of
climate change.  As any economist, game-theory expert, or parent will
attest: to see real change, the incentives must be altered.  The
incentive structure above illustrates actions which can be taken by
many parties which work in concert with one another to relegate
\acp{kasat} to the same type of strategic deterrence as nuclear
weapons.  This game cannot be won, but at least a draw can be forced.


\subsection*{Efficacy: Disaggregation}
While this mitigation strategy is arguably the weakest, the cost of
execution is so low that its effective \ac{roi} is quite high.
Current \acp{kasat} are expensive singular assets similar to (if not
deriving directly from) \acp{abms}.  This economic model of large,
expensive, and highly sophisticated assets can be overcome by
transitioning away from large and sophisticated national security
assets, and towards resilient constellations.  This approach holds
significant potential for economic benefits both to departments of
defense, as well as to commercial industry.\cite{big-risks} The
various innovations by industry to decrease the cost of launching and
maneuvering a commercial constellation will also serve to decrease the
cost of \ac{kasat} methods, however, which will likely result in a
continual cat-and-mouse game.

Other strategies exist, but none as cost effective as transitioning to
resilient constellations of ``good enough'' satellites.  The laws of
physics suggest that many assets such as \ac{hsi} satellites on
\ac{isr} missions will remain large and expensive for some time to
come.

\subsection*{Availability: Alternative Weapons}
For the mission of permanently eliminating an adversary's access to a
spacecraft of strategic importance, few options exist beyond
\acp{kasat}.  If, in times of war, military decision-makers reach for
an effective weapon, it would be better if that weapon did not cause
widespread destruction.  It is critical, then, that all militaries
with access to \acp{kasat} also develop \acp{safe}.  This maxim is
especially true for irresponsible actors who have already demonstrated
a clear lack of regard for generation of long-lived orbital debris.
This mitigation strategy is best summarized as equipping a child with
a stick if they are already in possession of a stick of dynamite.  You
can't stop them from using the stick of dynamite, but perhaps with the
right set incentives, you can guide them into using the stick instead.

\subsection*{Governance: Internal and External}
For decision-makers to select the newly developed weapon which does
not generate debris instead of the old tested weapon that does, they
will need incentive to do so.  This incentive can come in many forms
at all levels of the process. \ac{cnc} safeguards can be put in place
that govern military officers.  Civilian oversight can help ensure the
development of clear doctrine related to use of \acp{kasat}.  Global
response and escalation policies can be developed and published
ensuring that commanders know the military consequences of using
\acp{kasat}, even if they don't fully fathom the orbital consequences.
Usage-limiting treaties can be signed placing countries on quota
systems ensuring both the incentive to develop less-destructive
weapons and to be more judicious in their testing and use.  Instead of
simply lobbing foul language at autocratic leaders, physical response
policies can be put in place and enforced.

Ultimately, a combination of both internal education and structural
controls as well as external pressure and strategic threats backed up
by military force can provide the same kind of deterrence to use that
is found in nuclear weapons.

\section*{Conclusion}
Attempts to ban \acp{kasat} are as likely to succeed as attempts to
ban nuclear weapons; the approach is too familiar and too effective to
outright eliminate.  Even if military leaders wanted to make a
different choice, they have few effective alternatives.  Even if
alternatives were available, there is little in the way of controls,
governance, or real international consequences to their use.  The
incentive structure is clearly heavily in favor of the development,
deployment, and use of \acp{kasat}.

Much as our hominid ancestors learned, hitting something with a rock
works.  Until the laws of physics change, until the mathematics of
game theory are overcome, until the nature of humans is altered, until
science fiction writers can envision a future in which weapons and war
are not a part of life, kinetic attacks in orbit will continue to
exist and be a threat.  We have learned our lesson with nuclear
armaments, and use them primarily as a strategic deterrent.  While we
all might prefer to ban nuclear weapons outright on Earth or
debris-generating weapons in orbit, we are extremely unlikely to
succeed.  Our successes, as a species, have not come through
ideological purity and devotion to an ideal so much as through
pragmatic reality.  We aren't going to succeed at banning
\aclp{kasat}, so let's at least ensure that they stay on the shelf.
