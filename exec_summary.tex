\titlenote{EXECUTIVE SUMMARY}

\maketitle

%% These are fancyheader styles defined in the preamble.  We set the
%% general style, then the style for the first page.  We do the same
%% again for the paper itself.
\pagestyle{theRest}
\thispagestyle{firstPage}

%%
%% Yeah...I know...I should be using relative locations
\textblockorigin{1in}{\paperheight-.85in}

\begin{textblock*}{2.5in}(0in, 0in)
\tiny
{\noindent\copyright 2022 Harrison Caudill
All Rights Reserved
}
\end{textblock*}

\begin{textblock*}{2.5in}(5.5in, 0in)
\tiny
    {\noindent
      \hfill
      Revision:
      \input{rev}
      \hspace{1in}
    }
\end{textblock*}


\addcontentsline{toc}{section}{Executive Summary}

Among scientists and spacefaring entrepreneurs, \ac{asat} missiles,
which threaten humanity's future in orbit, are almost universally
despised.  That disgust for these weapons seems to have led to
multiple efforts to ban them (or at least ban their
testing)\footnote{See \S\ref{section::governance::prior}}.  Despite
nearly 80 years of history with nuclear armaments, the world is no
closer to banning them entirely; they are simply too valuable to
governments and militaries.  If \ac{asat} missiles are of utility to
militaries (and thus a risk to the future of humanity in orbit) then
attempts to ban them are likely to meet with similarly dismal results
and are a waste of time.  If \ac{asat} missiles are not of utility,
then attempts to ban them are unnecessary, and thus a waste of time.
Either way, attempting to ban \ac{asat} missiles is a waste of time
and opportunity.  Attempts to attenuate the odds of their use,
however, may yet succeed.

The fundamental concept of an \ac{asat} missile is relatively
straight-forward, and follows the general trend of weapons-development
throughout history: hit the target with something really hard that's
moving really quickly.  The primary reason the world wishes to ban
\ac{asat} missiles is because debris in orbit doesn't simply settle to
the ground where it can be swept up and discarded; it remains in orbit
and threatens other spacecraft.  Humans in orbit already require a
special type of shielding to protect them from collision with space
debris, and the \ac{iss} has already replaced several windows due to
collisions.  The relative speed of a head-on collision in \ac{leo} is
approximately 36,000mph at that speed, the two colliding entities
would both effectively detonate into puffs of debris.  Much like an
uncontrolled fission reaction in a nuclear bomb, the potential for
exponential growth of debris rests tenuously upon the current status,
distribution, and management of what is already in orbit.  That
exponential ``runaway'' increase of debris, known as ``Kessler
Syndrome'', is what the world wishes to avoid.  Sadly, it is already
too late to ``avoid'', as it has already started; \ac{asat} missiles
simply make the problem more acute more quickly.

If \ac{asat} missiles are bad and cannot be banned, what \emph{can} be
done about the problem?  This paper offers a proposed answer to that
question by first examining the incentive structures around the
development, testing, deployment, and use of \ac{asat} missiles and
then illustrating actions which can be taken to attenuate those
incentives.

\section*{Incentives}
Much like root structures for plants follow the availability of water,
and corporations follow the availability of profit, humans can usually
be trusted to follow their own personal incentives.  To attenuate the
risk of \ac{asat} missiles, one should attenuate the incentives to use
them, and provide disincentives at the same time.  This paper examines
the incentive structures around \ac{asat} missiles and uses that
examination to provide recommendations for how to attenuate their risk
to humanity's future among the stars.


\subsection*{Efficacy: They Get the Job Done}

The ``job'' in this context, is to permanently and completely
eliminate an adversary's strategic capability.  For example,
destruction of a number of SBIRS satellites could cripple the US'
ability to monitor for \acl{icbm} launches; regaining that capability
would require that the US build and launch new satellites.  A kinetic
attack on enemy spacecraft causing physical destruction of the target
will always get the job done.  Much as we find the terrestrial
warfighting domain, countermeasures can be developed and deployed; but
the fundamental approach still works and will continue to do so for
the foreseeable future.


\subsection*{Availability: Few Viable Alternatives Exist}

A distinction should be drawn between a \ac{kasat} a \ac{safe}.  A
\acf{kasat} is a debris-generating weapon, while a \ac{safe} is not.
Extensive use of \acp{kasat} would generate large quantities of
problematic debris in orbit, whereas widespread use of \acp{safe}
would not result in a significant increase in the quantity of orbital
debris.

If the sought-after qualities of a weapons system are that the
disablement be complete, and the effects be permanent, then no
alternatives are known to be deployable at this time.  Jamming
communications from ground-terminals, for example, is not permanent as
it requires that the target spacecraft be within range of the
transmitter.  Similarly, using \acp{laser} to physically damage an
optical sensor is not complete as the non-optical systems would remain
functional.

If the mission is to completely, and permanently take a strategic
orbital asset off the board in time of war: \acp{kasat} are the only
widely-available and easily-understood method.


\subsection*{Governance: Few Internal or External Controls}

\ac{cnc} safeguards against accidental deployment of nuclear weapons
are well publicized, well studied, and at least somewhat understood.
By contrast, no \ac{cnc} safeguards against inadvertent launch of a
\ac{kasat} are known.  While the world may generally condemn the
deployment of \acp{kasat}, how much weight does that condemnation
really carry?  Russia, for example, was widely condemned for their
2021 \ac{kasat} test and also for their 2022 invasion of Ukraine.  It
seems likely that they would have known in advance that such
condemnation would occur, and they took those steps anyway.  Unlike
nuclear weapons, response and escalation policies do not seem to
exist and certaily aren't enforced.


\section*{Recommended Mitigations}

Strong words and fervent hopes are as likely to affect the outcome of
deployment of \acp{kasat} as they have at altering the course of
climate change.  As any economist, game-theory expert, or parent will
attest: to see real change, the incentives must be altered.  The
incentive structure above provides actions which can be taken by many
parties which work in concert with one another to relegate \acp{kasat}
to the same type of strategic deterrence as nuclear weapons.  We
cannot win this game, but we can force a draw.


\subsection*{Efficacy: Divide and Hide}
While this mitigation strategy is arguably the weakest, the cost of
execution is so low that its effective \ac{roi} is quite high.
Current \acp{kasat} are expensive singular assets similar to (if not
deriving directly from) \acp{abms}.  This economic model of large,
expensive, and highly sophisticated assets can be overcome by
transitioning away from large and sophisticated national security
assets, and towards resilient constellations.  This approach holds
significant potential for economic benefits both to departments of
defense, as well as to commercial industry.\cite{big-risks} The
various innovations by industry to decrease the cost of launching and
maneuvering a commercial constellation will also serve to decrease the
cost of \ac{kasat} methods, however, which will likely make this a
continual cat-and-mouse game.

Other strategies exist, but none as cost effective as transitioning to
resilient constellations of ``good enough'' satellites.  The laws of
physics suggest that many assets such as \ac{hsi}
satellites\footnote{Quantum ghost imaging has the potential to
miniaturize these specific assets.} on \ac{isr} missions will remain
large and expensive for some time to come.

\subsection*{Availability: Alternative Weapons}
For the mission of permanently eliminating an adversary's access to a
spacecraft of strategic importance, few options exist beyond
\acp{kasat}.  If, in times of war, military decision-makers reach for
an effective weapon, it would be better if that weapon did not cause
widespread destruction.  It is critical, then, that all militaries
with access to \acp{kasat} also develop \acp{safe}.  This maxim is
especially true for irresponsible actors who have already demonstrated
a clear lack of regard for generation of long-lived orbital debris.
This mitigation strategy is best summarized as equipping a child with
a stick if they are already in possession of a stick of dynamite.  You
can't stop them from using the stick of dynamite, but perhaps with the
right set incentives, you can entice them to use the stick instead.

\subsection*{Governance: Internal and External}
For decision-makers to select the newly developed weapon which does
not generate debris instead of the old tested weapon that does, they
will need incentive to do so.  This incentive can come in many forms
at all levels of the process. \ac{cnc} safeguards can be put in place
that govern military officers.  Civilian oversight can help ensure the
development of clear doctrine related to use of \acp{kasat}.  Global
response and escalation policies can be developed and published
ensuring that commanders know the military consequences of using
\acp{kasat}, even if they don't fully fathom the orbital consequences.
Usage-limiting treaties can be signed placing countries on quota
systems ensuring both the incentive to develop less-destructive
weapons and to be more judicious in their testing and use.  Instead of
simply lobbing foul language at autocratic leaders, physical response
policies can be put in place and enforced.

Ultimately, a combination of both internal education and structural
controls can be enacted as well as external pressure and strategic
threats backed up by military force can provide the same kind of
deterrence to use that is found in nuclear weapons.

\section*{Conclusion}
Attempts to ban \acp{kasat} are as likely to succeed as attempts to
ban nuclear weapons; they are simply too fundamentally valuable to
militaries to outright eliminate.  The approach is effective, giving
military leaders an incentive to use them.  Even if those same leaders
wanted to make a different choice, they have few effective
alternatives.  Even if alternatives were available, there is little in
the way of controls, governance, or real international consequences to
their use.  The incentive structure is clearly heavily in favor of the
development, deployment, and use of \acp{kasat}.

Much as our hominid ancestors learned, hitting something with a rock
works.  Until the laws of physics change, until the mathematics of
game theory are overcome, until the nature of humans is altered, until
science fiction writers can envision a future in which weapons and war
are not a part of life, kinetic attacks in orbit will continue to
exist and be a threat.  Our successes, as a species, have not come
through ideological purity and devotion to an ideal so much as through
pragmatic reality.  We aren't going to succeed at banning
\aclp{kasat}, so let's at least ensure that they stay on the shelf.
