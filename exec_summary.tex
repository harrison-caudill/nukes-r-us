\titlenote{EXECUTIVE SUMMARY}

\maketitle

%% These are fancyheader styles defined in the preamble.  We set the
%% general style, then the style for the first page.  We do the same
%% again for the paper itself.
\pagestyle{theRest}
\thispagestyle{firstPage}

%%
%% Yeah...I know...I should be using relative locations
\textblockorigin{1in}{\paperheight-.85in}

\begin{textblock*}{2.5in}(0in, 0in)
\tiny
{\noindent\copyright 2022 Harrison Caudill
All Rights Reserved
}
\end{textblock*}

\begin{textblock*}{2.5in}(5.5in, 0in)
\tiny
    {\noindent
      \hfill
      Revision:
      \input{rev}
      \hspace{1in}
    }
\end{textblock*}


\addcontentsline{toc}{section}{Executive Summary}

Among scientists and spacefaring entrepreneurs, antisatellite
missiles, which threaten humanity's future in orbit, are almost
universally despised.  That disgust for these weapons seems to have
led to multiple efforts to ban them (or at least ban their testing).
It seems exceedingly unlikely that those efforts will succeed as
intended; antisatellite missiles are simply too valuable as strategic
military assets.  A similar pattern of incentive structures can be
found with nuclear weapons.  It seems extraordinarily unlikely that
the world will agree to eliminate strategic weapons capabilities, but
may yet agree to governing their testing, management, and usage.  By
examining the incentive structures surrounding antisatellite missiles,
specific and tractable actions may be found that could reduce their
harm.

The fundamental concept of an ASAT missile is relatively
straight-forward, and follows the general trend of weapons-development
throughout history: hit the target with something really hard that's
moving really quickly.  While that strategy may work well for
contained warfare in terrestrial domains, the consequences in orbit
are significantly more severe.  The resulting debris cloud not only
has the potential to persist for millennia it can also generate
yet-more debris by causing further collisions.  While the most
pernicious of terrestrial armaments (chemical and nuclear weapons) did
not see consequential usage until relatively late in human history
(WWI and WWII), warfare in space largely begins with KKVs and has not
progressed far beyond that point.  This absence of the
space-equivalent of conventional armaments, or ASATs that are
non-kinetic in nature, is highly problematic.  Additionally, the lack
of national and global governance of these weapons all but invites
problems.

\subsection*{Incentives}

There are three main aspects of the incentive structure nations face
when developing and deploying KKV ASATs.  While potential for ASAT
usage has been widely examined, this specific and unique
characterization of the incentive structure is instructive in that it
offers us tractable opportunities for mitigation.

\textbf{Efficacy:} ASAT missiles are effective at eliminating
strategic targets like surveillance systems or even other co-orbital
ASAT systems - it's a tool that gets the job done.

\textbf{Availability:} Currently there are few publicly-disclosed
and/or deployed options for a permanently disabling attack on a
spacecraft that do not generate debris - few other tools exist.

\textbf{Governance:} The command-and-control structures for strategic
ASAT systems are not generally known, civilian oversight, and societal
promise of retribution are all absent for ASAT systems - there is
little reason not to use the tool.

\subsection*{Mitigations}
From this incentive structure, mitigation strategies can be examined:

\textbf{1. Efficacy:} Few strategies exist for reducing the efficacy
of KKV systems and none have been publicly demonstrated to be
effective.  Deployed stealth satellites, for example, might decrease
the odds of detection by limiting RADAR cross-section, albedo, and
presented surface temperatures.  Much research has been done on this
topic as it pertains to avoiding ABM shields.  The approaches that
would likely be necessary would almost certainly be economically
and/or operationally infeasible, so it is assumed that deployed KKVs
are effective and will remain so.  If KKVs are always going to be
effective, then friendly and adversarial militaries alike must find
alternative weapons to fill that operational niche.

\textbf{2. Availability:} For the mission of permanently eliminating
an adversary's access to a spacecraft of strategic importance, few
options exist beyond kinetic antisatellite missiles.  If, in times of
war, military decision-makers reach for an effective weapon, it would
be better if that weapon did not cause widespread destruction.  It is
critical, then, that all militaries with KKV capabilities also develop
alternative weapons which accomplish the same mission, but without
generating the same amount of debris.  This maxim is especially true
for irresponsible actors who have already clearly demonstrated a lack
of regard for generation of long-lived orbital debris.  This
mitigation strategy is best summarized as equipping a child with a
stick if they are already in possession of a stick of dynamite.  You
can't stop them from using the stick of dynamite, but perhaps with the
right set incentives, you can entice them to use the stick instead.

\textbf{3. Governance:} For decision-makers to select the newly
developed weapon which does not generate debris instead of the old
tested weapon that does, they will need incentive to do so.  This
incentive can come in many forms at all levels of the process. CNC
(Command aNd Control) safeguards can be put in place.  Civilian
oversight can help ensure the development of clear doctrine related to
use of KKVs.  Global response and escalation policies can be developed
and published ensuring that commanders know the military consequences
of using KKVs, even if they don't fully fathom the orbital
consequences.  Usage-limiting treaties can be signed placing countries
on quota systems ensuring both the incentive to develop
less-destructive weapons and to be more judicious in their testing and
use.

\subsection*{Recommendations}

Encourage belligerents (especially Russia/China but also US/India) to
transition military strategies away from KKVs and towards SAFEKILLs.

Stop trying to ban ASAT missiles, and start trying to bring useful
governance such as per-nation quotas on space debris, CNC safeguards,
transparency of stockpiles, use-of-force doctrines, de-escalation
hotlines, etc.

\subsection*{Conclusion}

An examination of the incentive structures surrounding kinetic ASAT
weapons shows us a path by which their odds of use and negative
impacts may be diminished.  The human race may never rid itself of
these weapons, and certainly won't rid itself of war, but there may
yet be a path to decrease the potential consequences of KKVs.  By
refocusing our efforts on mitigation, rather than elimination, we may
yet do some good.
