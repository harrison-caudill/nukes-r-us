\titlenote{EXECUTIVE SUMMARY}

\maketitle

%% These are fancyheader styles defined in the preamble.  We set the
%% general style, then the style for the first page.  We do the same
%% again for the paper itself.
\pagestyle{execSummary}
\thispagestyle{firstPage}

%%
%% Yeah...I know...I should be using relative locations
\textblockorigin{1in}{\paperheight-.85in}

\begin{textblock*}{2.5in}(0in, 0in)
\tiny
{\noindent\copyright 2022 Harrison Caudill
All Rights Reserved
}
\end{textblock*}

\begin{textblock*}{2.5in}(5.5in, 0in)
\tiny
    {\noindent
      \hfill
      Revision:
      \input{rev}
      \hspace{1in}
    }
\end{textblock*}


\addcontentsline{toc}{section}{Executive Summary}

Since it was first discovered that a large stick could be employed by
one primate to beat another primate to death, the application of
kinetic energy in war has been a mainstay.  In the terrestrial domain,
that means bazookas and bullets, missiles and mines; in the orbital
domain, that means \ac{asat} missiles.  While we do have some
fundamentally new techniques such as cyber warfare\cite{big-risks},
and while the precision and quantity of application has changed, most
approaches in warfare, even in orbit, come down to application of
kinetic energy.\cite{brian} The general consensus among space industry
professionals and scientists is that \ac{asat} missiles should be
banned.\cite{uocs-no-likey}\cite{carnegie-no-likey} The debris, or
``Space Junk'' generated by a successful impact can easily cause
decades of chaos.\cite{osa-debris} On numerous occasions, attempts
have been made to ban these weapons\cite{early-to-call} with the most
recent being a motion before the UN by the US to ban a specific type
of \ac{asat} missile.\cite{us-asat-me-not} Since there are multiple
types of \ac{asat} missiles, and even more types of \ac{asat} weapons
that generate debris, this proposal addresses only a fraction of the
overall risk profile.\cite{brian} Even a ban on that subset was
rejected by Russia and China.\cite{not-so-chinese} The application of
kinetic energy (i.e. hitting a target with a rock, a bullet, or
shrapnel) causing a target to break apart is highly effective and
therefore unlikely to ever be truly banned.  If, like nuclear weapons,
\acp{kasat} cannot be banned, then they should be managed.  This paper
examines the incentive structures associated with \acp{kasat} finding
concrete steps which can be taken to alter those incentives and reduce
the risk of catastrophic deployment of \acp{kasat}.

While nuclear weapons may exist in abundance, they largely go unused.
This solution may be far from perfect, but the arrangement of
strategic deterrence is at least effective at preventing a large-scale
nuclear war.  A similar equilibrium may be sought for \acp{kasat}.  It
might not be possible to eliminate them entirely, but if they can be
relegated to the station of strategic deterrent, then they too might
go unused.  At present, however, the incentives to deploy them are
strong, the governance and control of them weak, the alternatives are
practically non-existent, and the consequences of use are, at best,
unknown.  Recommendations listed below directly address these aspects
of the incentive structure.

\section*{\ac{asat} Missiles and \acfp{kasat}}

The potential for exponential growth of debris (known as ``Kessler
Syndrome'') rests tenuously upon the current status and management of
the materials already in orbit, and is what many seek to avoid.
Unfortunately, it is already too late to ``avoid'' Kessler Syndrome,
as it has already started.  Kessler et al predicted that the
exponential increase would begin once a critical density was reached
and that point was reached some time ago; the exponential increase has
already begun.\cite[p14]{kessler-reunion} \ac{asat} missiles cannot
``cause'' Kessler Syndrome, but they can make it more acute more
quickly.\cite[p10]{kessler-reunion}

The fundamental concept of a \ac{kasat} is straight-forward and
consistent with most weapons development throughout history: apply
more kinetic energy than the target can safely absorb or deflect.
\ac{asat} missiles hit the target with a \ac{kkv}, terrestrial kinetic
weapons such as guns use bullets, etc.  The primary reason the world
wishes to ban \ac{asat} missiles is the resulting debris that doesn't
settle safely to the ground but instead remains in orbit and threatens
other spacecraft.  Humans in orbit already require a special type of
shielding to protect them from collision with space
debris.\cite{whipple-me-spacecraft} The relative speed of a head-on
collision in \ac{leo} is approximately 35,000mph.  At that speed, even
a fleck of paint is enough to irreparably damage a spacecraft
window.\cite{iss-throwing-stones}

\ac{asat} missiles come in two flavors; the \acf{coasat} and the more
common \acf{dasat}.\cite{brian} \acp{coasat} are already in orbit
allowing them to attack at any moment, but usually requiring half an
orbital period before impact.\cite{brian} \acp{dasat} are launched
from the ground allowing for rapid interception, but with the caveat
that the target must be nearby.\cite{asat-history} \ac{asat} missiles
may be the most common type of \ac{kasat}, but they are by no means
alone.  The Soviet Union, for example, launched a satellite with a
machine gun, and also detonated spacecraft near their targets like
orbital mortar rounds.\cite{brian}

While the world's attention may be on \ac{asat} missiles, it would be
more appropriate to consider the entire category of kinetic weapons.
Any weapons which, through application of kinetic energy generate
debris in orbit will be referred to as \acp{kasat}.  Weapons which do
not will be referred to as \acp{sasat}.  This distinction is rarely
discussed; the aforementioned motion at the UN, for example, does not
even attempt to ban the other common type of \ac{asat} missile
(\acp{coasat}), let alone all \acp{kasat}.\cite{un-asat-me-not}

\section*{Incentives}

\ac{asat} missiles are unlikely to be banned, because they accomplish
an important objective: they destroy strategic orbital assets.
Examining the incentives associated with these weapons provides
insights into ways that their impacts can be minimized.

\subsection*{Efficacy: They Get the Job Done}

Application of kinetic energy is a weapons principle that stretches
back to the very first hominid ancestor using a stick.  It is a
principle so firmly rooted in physical laws that it is all but
guaranteed to be an effective approach well into the future.

While, the \ac{dasat} in its present form may have seen the peak of
its utility as the world transitions most orbital assets away from
large single targets and towards resilient constellations, the
fundamental principle of applying kinetic energy has not lost utility,
nor is it likely to.  Additionally, several physical laws suggest that
there will continue to be some number of large and expensive
satellites.


\subsection*{Availability: Few Viable Alternatives Exist}

Two strategic attributes exist for \acp{kasat}: their effects are
permanent and their destructive power is complete.  Jamming
communications from ground-terminals, for example, is not permanent as
it requires that the target spacecraft be within range of the
transmitter.  Similarly, using LASERs to physically damage an optical
sensor is not complete as the non-optical systems would remain
functional.  If the mission is to completely, and permanently take a
strategic orbital asset off the board in time of war: \acp{kasat} are
the only widely-available and easily-understood method.\cite{brian}


\subsection*{Governance: Few Internal or External Controls}

\ac{cnc} safeguards against improper deployment of nuclear weapons are
well publicized, well studied, and at least somewhat understood.  By
contrast, no \ac{cnc} safeguards against improper launch of a
\ac{kasat} are known.

While the world may generally condemn the deployment of \acp{kasat},
how much weight does that condemnation really carry?  Judging by
Russia's 2021 ``test'' of the Nudol \ac{dasat}, and their response to
ensuing condemnation, the answer is ``not much''.  Unlike nuclear
weapons, response policies do not seem to exist.


\section*{Recommended Mitigations}

Strong words and fervent hopes are as likely to affect the deployment
of \acp{kasat} as they are to alter the course of climate change.  To
see real change, the incentives must be altered.  Specific actions can
be taken to alter several aspects of the incentive structure above.
These actions can work in concert with one another to relegate
\acp{kasat} to the same type of strategic deterrence as nuclear
weapons.  There is an argument that an unofficial policy of strategic
deterrence already exists, but none of the same structure that one
finds for nuclear weapons seems to exist for \acp{kasat}.  This game
cannot be won by any party, but at least a draw can be forced.


\subsection*{Efficacy: Disaggregation}
While this mitigation strategy is arguably the weakest, the economic
benefits of adoption are expected to be high.  The most common
\acp{kasat} at present (\acp{dasat}) are expensive assets similar to
(if not derived directly from) \acp{abms}.\cite[pxiii]{brian} This
economic model of large, expensive, and highly sophisticated weapons
can be overcome by transitioning away from large and sophisticated
spacecraft, and towards resilient constellations.  However, the
various innovations by industry to decrease the cost of launching and
maneuvering a commercial constellation will also serve to decrease the
cost of \ac{kasat} methods, which will likely result in a continual
cat-and-mouse game.  Since China seems to be employing this strategy
already, one could argue that the strategic deterrence effect of
American \acp{dasat} is waning already.

Other strategies exist, but none as cost effective as transitioning to
resilient constellations of ``good enough'' satellites.  Even so, the
laws of physics suggest that many assets, such as optical imaging
satellites, will remain large and expensive for some time to come.

\subsection*{Availability: Alternative Weapons}
For the mission of permanently eliminating an adversary's access to a
spacecraft of strategic importance, few options exist beyond
\acp{kasat}.  If, in times of war, military decision-makers reach for
an effective weapon, it would be better if that weapon did not cause
widespread destruction.  It is critical, then, that all militaries
with access to \acp{kasat} also develop \acp{sasat}.  This maxim is
especially true for irresponsible actors who have already demonstrated
a clear lack of regard for generation of long-lived orbital debris.
This mitigation strategy is best summarized as equipping a child with
a stick if they are already in possession of a stick of dynamite.  You
can't stop them from using the stick of dynamite, but perhaps with the
right set incentives, you can guide them into using the stick instead.

\subsection*{Governance: Internal and External}
For decision-makers to select the newly developed weapon which does
not generate debris instead of the old tested weapon that does, they
will need incentive to do so.  This incentive can come in many forms
at all levels of the process. \ac{cnc} safeguards can be put in place
that govern military officers.  Civilian oversight can help ensure the
development of clear doctrine related to use of \acp{kasat}.  Global
response and escalation policies can be developed and published
ensuring that commanders know the military consequences of using
\acp{kasat}, even if they don't fully fathom the orbital consequences.
De-escalation hotlines can be established.  Usage and testing
limitations can be enacted placing countries on quota systems ensuring
both the incentive to develop less-destructive weapons and to be more
judicious in their testing and use.

\section*{Conclusion}
Attempts to ban \acp{kasat} are as likely to succeed as attempts to
ban nuclear weapons; the approach is too familiar and too effective to
outright eliminate.  Even if military leaders wanted to make a
different choice, they have few effective alternatives.  Even if
alternatives were available, there is little in the way of controls,
governance, or real international consequences to their use.  It is
clear that the incentive structure heavily favors the development,
deployment, and use of \acp{kasat}.

As has been known since the beginning of recorded history, hitting
something with a rock works.  Until the laws of physics change,
kinetic attacks in orbit will continue to exist and be a threat.
Banning them may be unlikely, but using lessons from governance of
nuclear weapons and global pollutants, they might be relegated to
strategic deterrents and become no more of a detriment to society than
an unfortunate use of money.  It may feel right to try to ban them,
but humanity's progress has not come through ideological purity so
much as through pragmatic reality.  We aren't going to succeed at
banning \aclp{kasat}, so let's at least ensure that they stay on the
shelf.
