\section{Recommendation: ASAT Governance}

The governance structures presented here are designed to be exemplary
in nature and to stimulate discussion.  These recommendations provide
a glimpse of a stable end-point where the incentives of parties
involved are balanced.  However, the potential paths to reach that
stable point are poorly understood.

Seeking agreement on reasonable governance structures seems, at best,
enormously difficult.  China is engaged in slave labor and preparing
to invade Taiwan, Russia has already invaded and committed war-crimes
in Ukraine after assasinating or imprisoning political rivals, and the
United States only narrowly avoided a successful coup attempt by one
of the two major political parties - navigating ASAT governance
agreements in this complicated diplomatic space is well beyond the
scope of this paper.  Further complicating matters is the question of
what ``civilian oversight'' might even mean in the context of an
autocratic regime such as China, Russia, or potentially even the USA
if any of the anticipated followup coup attempts were to succeed.  It
could be argued that an externally-applied deterrent would be more
effective against such a regime than an internally-accepted governance
structure.

\subsection{Prior Attempts at Governance}
In 20XX, the XXX-organization put forward a proposal to ban the testing of ASAT missiles.
In 2022, a new call was put forward to ban testing of ASAT missiles.
XXX: Have any of the nations with ASAT missiles endorsed this treaty?


\subsection{Command and Control (CNC) Safeguards}
Historically, those who have been capable of launching nuclear missiles have shown restraint at every level.  In 19XX, a false positive on radar theoretically should have led to the launch of nuclear weapons, but did not, because the officer in charge (Lt. Who'sItsFace) strongly suspected it was a false positive and exercised restraint.  In 1962, Kennedy and Khrushchev were able to de-escalate from the brink of nuclear war.  To this day, legal and operational safeguards are in place at every level in the United States to ensure that nuclear weapons are not utilized either accidentally, flippantly, or irresponsibly.  It is suspected that officers in Russia would revolt before permitting nuclear launches.(cite ditmore's thing)

Military doctrine surrounding the use of ASAT missiles should be published.  In the event that such doctrine is largely absent, it should be developed.  For example:

Escalation Policy: The circumstances under which it is considered reasonable to launch ASAT missiles should be articulated, and enforced.

Chain of Command: The set of officers and leaders with authority to order the launch of ASAT missiles should be clearly defined, and those officers/leaders should be well-trained in the consequences of their actions.

Target Selection: While it appears as though China tested one of their ASAT missiles for use against an American GPS satellite, such an attack would generate debris that is likely to live for millenia and endanger some of the most valuable capabilities that all citizens from all nations currently enjoy.  It might be reasonable, subject to an escalation policy, to avoid targets at higher orbital altitudes.

Attack Methodology: If, as seems likely, multiple attack vectors are available, the least destructive vector should be chosen.  For example, a head-on collision will generate a very different debris profile than a strike in the nadir (facing the center of the Earth) direction.

De-Escalation: Similar to nuclear de-escalation channels, a method of communicating with command personnel at various ASAT-armed adversaries should be created.

No-First-Strike: Similar to nuclear armaments, nations should pledge to not be the first to launch an ASAT missile.

Several CNC safeguards can be enacted without substantively harming the secrecy and efficacy of the ASAT systems.  Information about, for example, de-escalation can easily be made public while other information, such as the number of tungsten rods ejected as part of the attack or the effective radius of the kill can be withheld.  The United States, for example, regularly updates and publishes nuclear doctrine and de-escalation procedures.(XXX: What to cite for this one??? Ask foster.)

Testing Limitations
Starting with the premise that stockpiles will not be eliminated and testing will not end, the obvious followup question is how to effectively limit testing.  Any viable solution must meet certain criteria:

Verification: It should be possible to verify any claims regarding test results in as unambiguous a manner as possible.

Attribution: Any violations of the agreed limitations should be directly attributable to the offending nation.

Flexibility: The restrictions should still permit reasonable testing to ensure efficacy.

Safety: Undue risk to civil space exploration should be avoided.

\subsection{Verification \& Attribution}
If testing limitations are defined strictly in terms of observable quantities available in the public domain, then verification becomes a tractable problem.  If the observable metrics cannot be publicly disclosed, then enforceability becomes questionable as no public statements of infraction can be made.  If the only source of information for these metrics is from governments, then political bias can also interfere.  Ideally, the metrics should be verifiable by private third parties in multiple nations.


Restriction
Known to Governments?
Known to Public?
Verifiable by Private Industry?
Orbital Altitude of Test
x
x
x
Relative Velocity at Impact
x
x


Angle of Impact


x


Orbital Inclination
x
x
x
Latitude/Longitude of Test
x
x


Mass of the Target
x
x


Mass of the Interceptor
?




Suborbital Trajectory
x
x


Number of Pieces of Trackable Debris
x
x
x
Number of Pieces of Untrackable Debris








While private industry is well-poised to observe that an ASAT test has occurred by observing changes in trackable debris, it is unclear whether or not there is sufficient commercial monitoring of unannounced launches to be capable of attributing a launch to a specific nation.

\subsection{Flexibility and Safety}
Definitions of safety would likely involve concepts such as ``Odds of a subsequent collision within 10 years''.  However, determining such values can be quite difficult even for cooperative parties.(citation needed???)  Since restricting complicated quantities such as that would eliminate the verifiability requirement, proxy values will likely be required instead.

If a numerical model can be agreed upon, and a theorem-proven implementation constructed then it could be an excellent method of limiting testing.  Alternatively, a more brute-force approach can be utilized.  For example, the total number of pieces of trackable debris can be restricted.  In both cases, a ``quota'' can be set.  If a numerical model for probability of collision is employed then the quota can be expressed as a probability.  If the restriction is on the number of pieces of trackable debris, then that number can represent the national quota.

The naive model present in the Gabby software package(cite github) has been shown to work reasonably well at predicting the overall status of debris clouds.  Using this model, several scenarios were run and it was found that the overall amount of trackable debris in orbit would be quite manageable under responsible testing conditions.  For example, as shown in figure 42 in Jonn's removed Appendix, the total number of pieces of trackable debris in orbit remains below 2,000 even if 3 Mission Shakti style events were to happen at random times each year.  Since that number of pieces of trackable debris represents a small fraction of the total number, an argument could be made that it is ``manageable''.  However, it should be noted that further research and consultation with existing experts is required to guard against unseen threats, such as untrackable debris.  Current models predict the mean time between natural collisions between existing objects in orbit to be measured in years.(cite...sh** where was that figure...it was super handy...)  Since the equilibrium amount of debris caused by this testing regimen would not appreciably change those circumstances, it seems reasonable, if naive, to assume that it would not accelerate the onset of Kessler Syndrome by much.

Ultimately, if each nation is on some form of impact quota (be it number of pieces of debris, or some other universally accepted computation, or ...), then it provides the incentives necessary to develop ASAT systems which minimize that impact.  They are incentivized to seek out ways to clean up their own debris.  They are incentivized to test at low orbital altitude.  They are incentivized to behave far more responsibly.

\subsection{Initial Conditions and Grandfathering}
Since Russia and China are likely already beyond any sensible debris quota, an argument could be made that those nations should be, at least partially, grandfathered in.  At the same time, however, Russia and China have both executed ASAT tests well after knowledge of the perils of Kessler Syndrome became widespread.  It seems reasonable to count any tests performed after the Kessler Divide (either after the publication of Kessler's seminal work or after sufficient mass was launched into orbit to initiate Kessler Syndrome) against their national quota.  In some sense, they knew better at the time, and did it anyway.  The obvious impediment, naturally, is that few nations have the wisdom, strength, and self-assurance to admit when they were wrong and to productively move forward.

\subsection{Enforcement}
In addition to the slew of traditional enforcement mechanisms for treaty violations, such as sanctions, NATO recently announced that there would likely be a military response if Russia employed nuclear weapons in Ukraine.  Since any military response must enjoy the popular support of those who bring power to the decision-makers, it is deeply unclear if use of ASAT missiles would be met with that conviction.

Remuneration for damaged spacecraft could be considered, but is generally problematic.  To request remuneration, one must first provide evidence that the offending party was responsible.  In the case of debris from an ASAT missile launch that means that a piece of trackable debris must then collide with an active spacecraft for there to be economic damages.  If the spacecraft is inactive, then there are no damages to consider.  On the other hand, if the spacecraft is still functioning then it likely has the ability to maneuver and avoid the collision in the first place.  Also, if the piece of debris causing the damage is untracked, then it also cannot be attributed to any specific nation.  That leaves damages from tracked pieces of debris to spacecraft that are still functioning but unable to maneuver to avoid the collision.  This scenario is expected to be an edge case.

