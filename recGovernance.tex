\section{Recommendation: ASAT Governance}
\label{section::governance}

The governance structures presented here are designed to be exemplary
in nature and to stimulate discussion.  These recommendations provide
a glimpse of stable end-points where the incentives of parties
involved are balanced.  However, the potential paths to reach those
stable points are beyond the scope of this publication.

Two parallels present themselves for governance structures: nuclear
weapons and environmental regulations.  Space debris is, quite
rightly, viewed as a pollutant.  \acp{kasat} are also strategic
weapons.  Historical precedent and learned lessons from prior
successes and failures such as the Montreal Protocol, Paris Accords,
and various nuclear test ban treaties may help inform future discourse
and actions.


\subsection{Recent Attempts at Governance}
\label{section::governance::prior}
\begin{itemize}

\item FIXME: Mine this source to catalog prior attempts:
  \cite{early-to-call}

\item In April, 2021, Nivedita Raju with the Stockholm International
  Peace Research Institute authored a
  \href{https://www.sipri.org/sites/default/files/2021-04/eunpdc_no_74.pdf}{
    proposal to ban ASAT testing}.\cite{eu-ban}

\item In April, 2022, U.S. Vice President Kamala Harris
  \href{https://www.youtube.com/watch?v=ZANjCHvSEpM}{announced} a
  commitment not to conduct further destructuve \ac{dasat}
  tests.\cite{kamala-pinky-swears} It should be noted that the
  commitment ``...not to conduct destructive direct-ascent
  antisatellite missile testing.'' is non-binding and seems like it
  would not have applied to the 2008 destruction of US 193 by an SM-3
  \ac{dasat}.

\item In September, 2022, the United States Mission to the UN
  submitted a
  \href{https://uploads.mwp.mprod.getusinfo.com/uploads/sites/25/2022/09/US-ASAT-Documents-1-1.pdf}{proposed
    resolution} banning the destructive testing of \ac{dasat}
  missiles.\cite{un-asat-me-not} It should be noted that the
  resolution only bans \acpl{dasat}, and does not address other
  \acpl{kasat} such as \acpl{co-asat}.\cite{un-asat-me-not} It also
  does not attempt to ban \acp{abm} interceptors.\cite{un-asat-me-not}

\end{itemize}


None of the prior attempts at banning appear to have gained much
traction with the UN Security Council, nor does the current proposal
appear likely to pass.  Furthermore, even if the present proposal did
pass, it contains far too many loopholes to be an effective ban on
\acp{kasat}.

\subsection{Command and Control (CNC) Safeguards}
Historically, those who have been capable of launching nuclear
missiles have shown restraint at every level.  In 19XX, a false
positive on radar theoretically should have led to the launch of
nuclear weapons, but did not, because the officer in charge
(Lt. Who'sItsFace) strongly suspected it was a false positive and
exercised restraint.  In 1962, Kennedy and Khrushchev were able to
de-escalate from the brink of nuclear war.  To this day, legal and
operational safeguards are in place at every level in the United
States to ensure that nuclear weapons are not utilized either
accidentally, flippantly, or irresponsibly.  It is suspected that
officers in Russia would revolt before permitting nuclear
launches.(cite ditmore's thing)

Military doctrine surrounding the use of ASAT missiles should be
published.  In the event that such doctrine is largely absent, it
should be developed.  For example:

\textbf{Escalation Policy:} The circumstances under which it is
considered reasonable to launch ASAT missiles should be articulated,
and enforced.

\textbf{Chain of Command:} The set of officers and leaders with
authority to order the launch of ASAT missiles should be clearly
defined, and those officers/leaders should be well-trained in the
consequences of their actions.

\textbf{Target Selection:} While it appears as though China tested one
of their ASAT missiles for use against an American GPS satellite, such
an attack would generate debris that is likely to live for millenia
and endanger some of the most valuable capabilities that all citizens
from all nations currently enjoy.  It might be reasonable, subject to
an escalation policy, to avoid targets at higher orbital altitudes.

\textbf{Attack Methodology:} If, as seems likely, multiple attack
vectors are available, the least destructive vector should be chosen.
For example, a head-on collision will generate a very different debris
profile than a strike in the nadir (facing the center of the Earth)
direction.

\textbf{De-Escalation:} Similar to nuclear de-escalation channels, a
method of communicating with command personnel at various ASAT-armed
adversaries should be created.

\textbf{No-First-Strike:} Similar to nuclear armaments, nations should
pledge to not be the first to launch an ASAT missile.

Several CNC safeguards can be enacted without substantively harming
the secrecy and efficacy of the ASAT systems.  Information about, for
example, de-escalation can easily be made public while other
information, such as the number of tungsten rods ejected as part of
the attack or the effective radius of the kill can be withheld.  The
United States, for example, regularly updates and publishes nuclear
doctrine and de-escalation procedures.(XXX: What to cite for this
one??? Ask foster.)

\subsection{Testing Limitations}
Starting with the premise that stockpiles will not be eliminated and
testing will not end, the obvious followup question is how to
effectively limit testing.  Any viable solution must meet certain
criteria:

\textbf{Verification:} It should be possible to verify any claims
regarding test results in as unambiguous a manner as possible.

\textbf{Attribution:} Any violations of the agreed limitations should
be directly attributable to the offending nation.

\textbf{Efficacy:} The restrictions should still permit reasonable
testing for \acp{kasat} development.

\textbf{Safety:} Undue risk to civil space exploration should be
avoided.

\subsubsection{Verification \& Attribution}
Any defined restriction would likely need to be derivable from
quantities that are observable not only to governments, but also to
private industry.  Absent some method of verification, effective
restrictions would be impossible.  Figure \ref{figure::observables}
lists several such attributes.

\begin{figure}
  \label{figure::observables}
  \centering
  \begin{tblr}[
      label = {tbl::gov::test},
    ]{%
      colspec = {Q[c,m]|Q[c,m,wd=1in]|Q[c,m,wd=1in]|Q[c,m,wd=1in]},
      row{odd} = {gray!25}, row{even} = {white},
      row{1} = {gray!65},
    }
    {\bf Restriction}
    & {\bf Known to Governments?}
    & {\bf Known to Public?}
    & {\bf Verifiable by Private Industry?}
    \\

    Orbital Altitude of Test & \derX{} & \derX{} & \derX{} \\
    Relative Velocity at Impact & \derX{} & \derX{} &  \\
    Angle of Impact & \derX{} & \derX{} &  \\
    Orbital Inclination & \derX{} & \derX{} & \derX{} \\
    Latitude/Longitude of Impact & \derX{} & \derX{} & \\
    Mass of the Target & \derX{} & \derX{} & \\
    Mass of the Interceptor & {\bf ?} & & \\
    Suborbital Trajectory & \derX{} & \derX{} & \\
    Number of Pieces of Trackable Debris & \derX{} & \derX{} & \derX{} \\
    Number of Pieces of Untrackable Debris & & & \\
  \end{tblr}
  \caption{Observable quantities which might be of use in defining
    testing and use limitations of \aclp{kasat}.}
\end{figure}

While private industry is well-poised to observe that a \ac{kasat}
test has occurred by observing changes in trackable debris, it is
unclear whether or not there is sufficient commercial monitoring of
unannounced launches to be capable of attributing a launch to a
specific nation (FIXME: Ask Brian about this one).

\subsubsection{Efficacy and Safety}
A good definition of acceptable use would likely involve concepts such
as ``Odds of a subsequent collision within 10 years'' or ``impact
debris expectancy''.\cite[p19]{italiano} However, determining those
values can be quite difficult even for cooperative
parties.\cite[p18]{italiano} While it is conceivable that a
formulation of these concepts could be developed and agreed-upon, that
prospect seems unlikely.  Proxy values (such as total number of pieces
of trackable debris above a certain size) might be more useful.

However, even ``simple'' quotas such as the total number of pieces of
trackable debris can still present a hopeless amount of complexity.
\ac{radar} systems are not universally consistent in their
capabilities\cite[needed?]{xxx}, nor are the complete capabilities of
any given nation necessarily known.\cite[needed?]{xxx} Different
\acp{radar} systems are effective at observing different sizes and
distributions of debris and as \ac{radar} systems become more and more
sophisticated, more debris will be cataloged.  If, for example, a
quota of 3,000 pieces of trackable debris were set one day when a
nation has only 1,500 pieces in orbit, then the next day a new
\ac{radar} system goes online and finds a new cohort of 2,000 pieces
attributable to their \ac{kasat} launch, does that new cohort count
against the nation's quota?  Even if a numerical model is agreed upon
which purports to predict the probability of a natural collision, its
output would likely still change when better knowledge of initial
conditions is made available and the same issue exists.

Still, several years of experience has suggested that the amount of
debris generated from a Mission Shakti style event is generally
``manageable''.  The naive model present in the
\href{https://github.com/harrison-caudill/gabby}{Gabby software
  package} has been shown to work reasonably well at predicting the
overall status of debris clouds.\cite{gabby} Using this model, several
scenarios were run and it was found that the overall amount of
trackable debris in orbit would be quite manageable under responsible
testing conditions.  For example, as shown in figure
\ref{figure::gabby::doomsday} in appendix \ref{appendix::gabby}, the
total number of pieces of trackable debris in orbit remains below
2,000 even if 3 Mission Shakti style events were to happen at random
times each year.  Since that number of pieces of trackable debris
represents a small fraction of the total number, an argument could be
made that it is ``manageable''.  However, it should be noted that
further research and consultation with existing experts is required to
guard against unseen threats, such as untrackable debris.  Current
models predict the mean time between natural collisions between
existing objects in orbit to be measured in years.\cite[sh** where was
  that figure...it was super handy...kessler-reunion maybe?]{xxx}
Since the equilibrium amount of debris caused by this testing regimen
would not appreciably change those circumstances, it seems reasonable,
if naive, to assume that it would not accelerate the development of
Kessler Syndrome by much.

Ultimately, if each nation is on some form of impact quota (be it
number of pieces of debris, or some other universally accepted
computation, or ...), then it provides the incentives necessary to
develop ASAT systems which minimize that impact.  They are
incentivized to seek out ways to clean up their own debris.  They are
incentivized to test at low orbital altitude.  They are incentivized
to behave far more responsibly.  They are incentivized to develope
\acp{safe}.

\subsection{Initial Conditions and Grandfathering}
Since Russia and China are likely already beyond any sensible debris
quota, an argument could be made that those nations should be, at
least partially, grandfathered in.  At the same time, however, Russia
and China have both executed ASAT tests well after knowledge of the
perils of Kessler Syndrome became widespread.  It seems reasonable to
count any tests performed after the Kessler Divide (either after the
publication of Kessler's seminal work or after sufficient mass was
launched into orbit to initiate Kessler Syndrome) against their
national quota.  In some sense, they knew better at the time, and did
it anyway.  The obvious impediment, naturally, is that few nations
have the wisdom, strength, and self-assurance to admit when they were
wrong and to productively move forward.

\subsection{Enforcement}
In addition to the slew of traditional enforcement mechanisms for
treaty violations, such as sanctions, NATO recently announced that
there would likely be a military response if Russia employed nuclear
weapons in Ukraine.  Since any military response must enjoy the
popular support of those who bring power to the decision-makers, it is
deeply unclear if use of ASAT missiles would be met with that
conviction.

Remuneration for damaged spacecraft could be considered, but is
generally problematic.  To request remuneration, one must first
provide evidence that the offending party was responsible.  In the
case of debris from an ASAT missile launch that means that a piece of
trackable debris must then collide with an active spacecraft for there
to be economic damages.  If the spacecraft is inactive, then there are
no damages to consider.  On the other hand, if the spacecraft is still
functioning then it likely has the ability to maneuver and avoid the
collision in the first place.  Furthermore, if the piece of debris
causing the damage is untracked, then it cannot be attributed to any
specific nation.  That leaves damages from tracked pieces of debris to
spacecraft that are still functioning but unable to maneuver to avoid
the collision.  This scenario is expected to be an edge case.


\subsection{Analogous Governance Systems}


The Montreal Protocol is an excellent example of international
cooperation to reduce a pollutant based upon numerical analysis and
restrictions.\cite{oh-canada} With continued international support and
consistent updates, the Monteral Protocol is on track to have reverted
damage to the Ozone Layer back to pre-industrial levels by
2060-2075.\cite{oh-canada}  The Paris Accords, on the other hand, have
so far been far less effective.\cite{lousy-paris}

Nuclear governance is also a clear analog.  The strategic nature of
the weapons in terrestrial wafare is closely related to that of
\acp{kasat} in the orbital domain.  Nuclear armaments have not been
banned, nor are they likely to be banned.  While much progress has
been made on banning nuclear testing, those tests still happen.  There
is also less of a necessity as numerical simulations have largely
supplanted physical tests.  It should be noted, however, that
numerical simulation proceeded hundreds of physical tests, thereby
allowing numerical models to be validated by real-world testing.  No
such long history of \ac{kasat} testing exists.
