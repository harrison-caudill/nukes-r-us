\section{Overview of \acf{asat} Weapons}

The simplest meaning of an \acf{asat} weapon, is a weapon designed to
debilitate or destroy a satellite.  The concept of a ``weapon'' can
cover quite a bit of territory.  Perhaps the most common form of
\ac{asat} weapon is the \ac{dasat} which propels a \ac{kkv} toward a
target spacecraft.  Upon impact, the \ac{kkv} and the target generally
transform into a puff of problematic debris.  ``\acs{asat} Weapon''
and ``\acs{asat} Missile'' are often used interchangeably.  However, a
distinction will need to be drawn between kinetic and non-kinetic
weapons.  More precisely, the distinction is between those weapons
that \emph{do} generate debris and those that \emph{do not}.  \ac{kin}
weapons which generate debris will be referred to as \acp{kasat}.
\ac{asat} weapons which do not generate debris will be referred to as
\acp{safe}.

\subsection{A Brief History of \ac{asat} Development}

This brief history of \ac{asat} weapons is included here for
convenience.  For a more thorough treatment of the topic, see
\href{https://swfound.org/media/207344/swf_global_counterspace_capabilities_2022.pdf}{Global
  Counterspace Capabilities} (source \cite{brian}) and
\href{https://www.ucsusa.org/sites/default/files/2019-09/a-history-of-ASAT-programs_lo-res.pdf}{A
  History of Anti-Satellite Programs} (source \cite{grego}).

Generation of orbital debris dates back to early explorations by the
Soviets in the late 1960's with the targeting of Kosmos
248.\cite[p05-01]{brian}.  As Laura Grego states:
\begin{blockquote}
  Countries recognized that satellites would have great military value
  even before any had been successfully launched into
  orbit.\cite[p2]{grego}
\end{blockquote}

This recognition provided an obvious and powerful incentive to find a
way to neutralize an adversary's orbital assets. That incentive
resulted in early testing and some early successes, with \acp{coasat}.
\acs{coasat} interceptors are satellites placed in orbit, which can
maneuver to intercept a target.  Early Soviet designs were intended to
intercept a \ac{leo} target within 1-2 orbits (or 1.5 to 3 hours).
\acp{dasat} on the other hand, launched from the ground or an
airplane, were technologically more difficult\cite[4]{grego} and were
expected to take longer due to the necessity of waiting for a launch
window \cite[3]{grego}.

Because satellites operate under similar conditions as incoming
\acp{icbm}, \acs{abms} have a great deal of overlap with \ac{asat}
missile systems.  The net result is that the line between an \ac{abms}
and a \ac{kasat} is all but indistinguishable.\footnote{The primary
missile expected to be utilized by the United states for \ac{dasat}
missions is a mid-course correction \ac{abms}
missile.\cite[p01-15]{brian}}

Testing moratoria\cite{xxx}, \ac{abm} treaties\cite{xxx}, and changing
economic\cite[boom]{xxx} and political\cite[fall of ussr]{xxx}
climates all played a role throughout the ensuing decades as interest
in \ac{asat} systems ebbed and flowed.\cite[xxx]{grego} As space
technologies advanced, so too did the arms race between the great
powers and ``...the United States and the Soviet Union/ Russia have
followed parallel and often mutually reinforcing paths toward the
militarization of space over the past 50 years.''\cite[p2]{grego}

More recently, interest in \ac{coasat} systems has begun to resurface,
with close-approach imaging systems \cite[the russian one maybe?]{xxx}
and physical manipulation.\cite[chinese one maybe?]{xxx} While a
transition away from \acp{gbi} intended for \ac{kin} missions is
considered beneficial, alternative weapons are still far from prolific
enough to be viable in conflict.

\subsection{\acfp{kasat}}

\acl{kin} is arguably the oldest and most basic of kill mechanisms to
which the human race has access: you hit the target with a
rock(et). There are a few key differences, as one might imagine,
between hitting something with a rock and employing a billion dollar
antisatellite missile.  The proverbial rock in an \ac{asat} missile,
usually referred to as the \acf{kkv}, contains some amount of
targeting and course-correction capabilities.\cite{sm3} In addition to
being a multimillion dollar smart rock, \acp{kkv} can impact the
target at many thousands of miles per hour (or at hypervelocity)
putting even the best baseball pitchers to shame.\cite[kkv impact
  simulation]{xxx} Satellites, it should be noted, do not generally
fare well with even low-speed collisions.\cite{whoopsies} Furthermore,
even battleships do not fare well against hunks of metal traveling at
many thousands of miles per hour \cite[any hypersonic paper]{xxx}.  At
hypervelocity speeds, even a fleck of paint can damage a
window.\cite[iss window]{xxx} A satellite can survive a hypervelocity
impact with a \ac{kkv} just about as well as one might
imagine.\cite[hypervelocity impact simulation]{xxx} The primary
objective of kinetic antisatellite missiles, much the same as the
primary objective of bullets, is to ensure that something moving very
very quickly physically collides with the target.\cite[needed?]{xxx}

There is one notable exception to this description of \ac{asat}
missiles.  In the 1960's, the Soviet Union conducted a series of
experiments with what are probably best described as ``flying
bombs''.\cite[xxx]{brian} In 19XX, for example, the Soviet Union
maneuvered satellite XXX such that it closely approached satellite
YYY.\cite[brian or grego]{xxx} They then detonated XXX causing the
debris from that detonation then collided with satellite YYY ending
its mission with extreme prejudice.\cite[xxx pick something]{xxx}
Other than a few historical curiosities, the public record does not
contain significant evidence of this approach being further employed
or even planned.  The nearest modern approximation is likely to be a
refueling system.

There are two ways that a \ac{kkv} can approach a target: either from
the ground, or from orbit.\cite[overview of asat missile paper
  needed]{xxx} If a missile is launched from the ground, it is
typically known as a \acf{dasat}.\cite[definition from brian?]{xxx} If
already in orbit and maneuvered for intercept, it is known as a
\acf{coasat}.\cite[definition from brian?]{xxx} At the moment, very
little evidence suggests that \ac{coasat} systems are widely deployed.
Generally speaking, modern \acp{coasat} are largely employed for
\ac{rpo} as will be discussed in greater detail
later.\cite[xxx]{brian}.

By process of elimination, that leaves \acp{dasat} as the primary
threat, and as will be discussed later, they are indeed a serious
threat.  \acp{dasat} behave much as one might expect.  In a torrent of
fire, they launch, and after reaching many thousands of miles per hour
collide with the target satellite turning the two into a puff of small
bits.\cite[kkv impact simulation]{xxx} The kinetic energy of impact is
sufficient to destroy any target spacecraft (or any battleship for
that matter given that the relative collision velocity is likely to be
circa than Mach 25 at sea level).\footnote{A 100kg missile traveling
at Mach 25 (at sea level on a normal day) has approximately 1 kiloton
of kinetic energy, which is about 7\% of a Hiroshima bomb and well
inside the working definition of a ``tactical'' nuclear weapon. XXX:
Put this computation into the spreadsheet of doom to be released with
the paper.} In a manner similar to how popular films depict the
devastating effects of deploying tungsten rods or small asteroids to
impact population centers, the kinetic energy of a \ac{dasat} missile
strike is certain to be devastating.  For context, spacecraft are
typically delicate enough that merely falling on their sides from a
sitting position is sufficient to greatly damage them.\cite{whoopsies}
Like the furious plasma of a nuclear blast or a fusion reactor, there
is no known defense against a successful collision.


\subsection{Debris Risk aka ``Kessler Syndrome''}

In 1978, while working with astronomical observations in the asteroid
belt, Donald Kessler wrote a paper which shifted the thinking of space
exploration near Earth: {\it Collision Frequency of Artificial
  Satellites: The Creation of a Debris Belt}.\cite{kessler-og} Known
eponymously ever since as ``Kessler Syndrome'', this paper outlines
the forces of statistics and physics that lead to the ``runaway''
creation of debris conjuring scenes from popular films such as Gravity
in which bits of metal and other detritus swarm in clouds so thick
that no spacecraft dare tread.  While the reality is much more mundane
it does still carry sufficient downside to cause serious concern.

Several misconceptions about Kessler Syndrome seem to exist, and a few
of them will be covered here for convenience.

\subsubsection{We Need to ``Avoid'' Kessler Syndrome}
While the worst of the effects predicted by Kessler et al are
certainly things we wish to avoid, it is already too late to ``avoid''
Kessler Syndrome entirely.  Kessler et al predict that once enough
mass is in orbit, then the exponential ``runaway'' breakup will happen
on its own.\cite[xxx]{kessler-reunion}.  As of their 2016 paper (XXX
link to paper?  list citation number?  title?), there is already
enough mass in orbit to trigger this
scenario.\cite[xxx]{kessler-reunion} Putting that another way, the
exponential growth predicted by Kessler et al is already underway,
though it might not ``feel'' like it because of the relatively small
number of collisions and active management.(XXX: citation even
necessary?  I'm stating my own interpretation of the cited paper's
conclusions) To avoid the worst of the effects, we already need, at
minimum, to actively manage the spacecraft and other detritus (such as
second-stage rocket engines) already in
orbit.\cite[xxx]{kessler-reunion}

\subsubsection{Mega Constellations are a Mega Problem}
Prior \ac{asat} missile test data illustrates how low altitude
collisions don't last long in orbit\cite[shakti analysis
  overview]{xxx}, while higher altitude collisions can last for
centuries.\cite[umm...]{xxx} Most of the planned mega constellations
involve low altitude orbits where the risk is greatly
lessened.\cite[cite my own work maybe?]{xxx} Quantifying the risk is a
difficult proposition at best, but qualifying is tractable.  Even if
widespread collisions were to occur in a mega constellation, the
effects would likely be relatively short compared with other
historical collisions such as the Iridium/Kosmos collision in
2009.\cite[my own youtube video maybe?]{xxx}

Since the Earth's atmosphere doesn't simply stop existing altogether
at any magical distance, but rather peeters off slowly as it gets
further from the surface, satellites do actually travel through some
amount of atmosphere and thus experience atmospheric drag.\cite[good
  survey of orbit decay]{xxx}.  At about 400km above the Earth's
surface, a \ac{leo} satellite will last up to about a
year.\footnote{There are several sources of variability ranging from
how much solar activity there is which can cause the atmosphere to
extend further outwards, to the ``Ballistic Coefficient'' of the
satellite (just like a feather will fall slower than a bullet in an
atmosphere, a satellite with a lot of area and a small amount of mass
will decay faster).}\cite[probably same survey]{xxx} At about 800km
above the Earth's surface, however, there is very very little
atmosphere and debris is expected to last for XXX.\cite[analysis of
  fengyun using the dread model]{xxx} While at least one constellation
is planned for operation at or below 250km\footnote{The identity of
this organization is held in confidence, but the intent is to utilize
high-capacity and high-isp thruster systems to stay in orbit.}, very
few organizations are doing the same and none are currently known to
the author be active.  After a collision, nearly all debris will pass
through an orbital altitude at or below the altitude of the point of
impact\footnote{This characterization is \emph{mostly} true, but
glosses over some effects such as oblateness of the earth and
gravitational nonuniformities.  For the purposes of the context in
which it is used, however, it should be reasonably accurate.}, so
collisions at lower orbits ensure that the resulting debris will pass
through a non-trivial amount of atmosphere at least once per orbit.

\subsubsection{Kessler Syndrome will Entirely Deny Access to Space}
Contrary to popular imagination, exponential increase of debris does
not necessarily deny us access to space.  Some orbits are more
affected by debris than others, some debris will naturally decay due
to atmospheric drag, and sometimes the probability of collision is
simply something that we can live with.  Increased quantities of
debris will increase the cost of activity in space by forcing more
avoidance maneuvers, decreasing expected service lifetimes, requiring
more Whipple shielding, etc. but is not expected to outright deny
access.

\subsubsection{\ac{asat} Missiles will ``Cause'' Kessler Syndrome}
As mentioned above, Kessler Syndrome is already underway and is
currently kept at bay by active management.  It may not ``feel'' like
an exponential explosion by virtue of being in the early stages where
the expected collision rate is quite small.  If Kessler Syndrome is
thought of as an exponential growth of debris, \ac{asat} missiles
effectively cause a ``time shift'' moving us further into the future.
The question is further complicated when considering the various
critical factors such as the orbital altitude of the collision,
relative angle, relative velocity, etc.  The author has, as yet, been
unable to find any publicly available analyses related to the amount
of time-shift associated with any given launch.  However, it is worth
noting that the more debris there is in orbit already, the less it
matters if we have one more launch.  Similar to a pileup on a freeway
during a snowstorm, the second collision may double the total number
of wrecked cars, but the 100th collision has little impact on the
overall picture.

\subsection{Previous Breakup Events}

Those not steeped in the mathematics and operational issues associated
with \ac{stm} may not fully appreciate the rancor associated with
events such as the 2007 destruction of Fengyun 1C in a Chinese
\ac{dasat} test.  Of all of the cases in recorded history where a
Breakup Event has occurred, in which two objects in Earth's orbit have
collided, a few stand out as singularly bad:

\begin{itemize}

\item The 2007 destruction of Fengyun 1C in a Chinese \ac{dasat} test.

\item The 2009 collision between Iridium 33 and Kosmos 2251

\item The 2021 destruction of Kosmos 1408 in a Russian \ac{dasat} ``test''.

\end{itemize}

Of the 4,379 pieces of tracked \ac{kin} debris that is still in orbit,
3,988 pieces are from those two \ac{dasat} missions representing
roughly 90\% of the tracked \ac{asat} debris.\cite[Table 5-1,
  p05-01]{brian} Because of the orbital altitude at which the Fengyun
1C destruction occurred, that debris is likely to be there for a very
long time.\cite{osa-debris} Roughly 1,500 pieces of debris remain
in orbit from the Iridium/Kosmos collision.

By definition, untracked debris is unknown.  While still dangerous,
many tens of thousands of pieces of untracked debris remain just so:
untracked.\cite[p05-01]{brian} Even a fleck of paint in orbit is
enough to damage windows on ISS.\cite[dig this one up]{xxx}.  If a
fleck of paint is sufficient to irrepabarably damage spacecraft
components, imagine what destruction could be unleashed by a
hypersonic collision resulting in a literal detonation of the
target.\cite[one of the hypersonic papers]{xxx}
