\section{Overview of Anti Satellite (ASAT) Weapons}

The simplest meaning of an Antisatellite (ASAT), is a weapon designed
to debilitate or destroy a satellite.  The concept of a ``weapon'' can
cover quite a bit of territory.  Perhaps the most common form of ASAT
is the ASAT missile which propels a Kinetic Kill Vehicle (KKV) toward
a target spacecraft.  Upon impact, the KKV and the target generally
transform into a puff of debris.  ASAT and ASAT missile are often used
interchangeably.  However, a distinction will need to be drawn between
kinetic ASAT weapons and non-kinetic weapons.  More precisely, the
distinction between those weapons that generate debris and those that
do not needs to be discussed.  For the sake of simplicity, kinetic
weapons which generate debris will be referred to as KKVs or either
Kinetic Kill Vehicles or Kessler Kill Vehicles.  Those ASAT weapons
which do not generate debris will be referred to as SAFEKILLs.  See
(grego) for a history of ASAT development.

\subsection{A Brief History of ASAT Development}

Generation of orbital debris dates back to early explorations by the
Soviets in the late 1960's with the targeting of Kosmos 248.(brian,
05-01).  As Lara Grego states, ``Countries recognized that satellites
would have great military value even before any had been successfully
launched into orbit.'', providing an obvious and powerful incentive to
find a way to neutralize an adversary's orbital assets.(grego, 2) That
incentive resulted in early testing and some early successes, with
co-orbital antisatellite (CO-ASAT) interceptors.  CO-ASAT interceptors
are satellites placed in orbit, which can maneuver to intercept a
target.  Early Soviet designs were intended to intercept a Low Earth
Orbit (LEO) target within 1-2 orbits (or 1.5 to 3 hours).  Direct
Ascent (DA-ASAT) methods on the other hand where the missile launches
from the ground or an airplane, were technologically more difficult
(grego, 4) and were expected to take longer due to the necessity of
waiting for a launch window (grego, 3).

Because satellites operate under similar conditions as incoming
Intercontinental Ballistic Missiles (ICBMs), Anti Ballistic Missile
(ABM) systems have a great deal of overlap with ASAT missile systems.
The net result is that the line between an ABM system and an ASAT
system is all but indistinguishable.  As time has progressed, most KKV
systems are ABM missiles.(XXX: Verify this statement and find a good
source, probably either grego or brian).

Testing moratoria, ABM treaties, and changing economic and political
climates all played a role throughout the ensuing decades as interest
in ASAT systems ebbed and flowed.(grego) As space technologies
advanced, so too did the arms race between the great powers and
``...the United States and the Soviet Union/ Russia have followed
parallel and often mutually reinforcing paths toward the
militarization of space over the past 50 years.''(grego, 2)

More recently, interest in CO-ASAT systems has begun to resurface,
with close-approach imaging systems (XXX: cite the russian one maybe?)
and physical manipulation.(XXX: cite the chinese one maybe?  Brian
covers some of this, as does lara.)  While a transition away from
ground-based interceptors intended for kinetic kill is considered
beneficial, these weapons are still far from prolific enough to be
considered a viable alternative in the face of of imminent conflict.

\subsection{Kinetic Kill Systems}

Kinetic kill is arguably the oldest and most basic of kill mechanisms
to which the human race has access: you hit the target with a
rock.(citation needed?)  There are a few key differences, as one might
imagine, between hitting something with a rock and employing a
multimillion dollar antisatellite missile.  The proverbial rock in an
ASAT missile, usually referred to as the Kinetic Kill Vehicle (KKV),
contains some amount of targeting and course-correction
capabilities.(cite the sm3 datasheet from wayback) In addition to
being a multimillion dollar smart rock, KKVs can impact the target at
many thousands of miles per hour (or at hypervelocity) putting even
the best baseball pitchers to shame.(cite XXX) Satellites, it should
be noted, do not generally fare well with even low-speed
collisions(cite the incident with lockheed).  Furthermore, even
battleships do not fare well against hunks of metal traveling at many
thousands of miles per hour (cite XXX).  At hypervelocity speeds, even
a fleck of paint can damage a window.(cite ISS needing new windows) A
satellite can survive a hypervelocity impact with a KKV just about as
well as one might imagine.(cite hypervelocity impact simulation
papers) The primary objective of kinetic antisatellite missiles, much
the same as the primary objective of bullets, is to ensure that
something moving very very quickly physically collides with the
target.(cite XXX) This object could be the primary KKV itself, or,
like the case of our ABM systems, a balloon filled with tungsten
rods.(cite XXX)

There is one notable exception to this description of antisatellite
missiles.  In the 1960's, the Soviet Union conducted a series of
experiments with what are probably best described as ``flying
bombs''.(cite brian) In 19XX, for example, the Soviet Union maneuvered
satellite XXX such that it closely approached satellite YYY.(cite
brian) They then detonated XXX causing the debris from that detonation
then collided with satellite YYY ending its mission with extreme
prejudice.(cite XXX) Other than a few historical curiosities, the
author is unaware of any significant evidence of this approach being
further employed or even planned.  The nearest modern approximation is
likely to be a refueling system.(opinion, no citation needed)

There are two ways that a KKV can approach a target: either from the
ground, or from orbit.(FIXME: how to cite?)  If a missile is launched
from the ground, it is typically known as a Direct Ascent
Antisatellite Missile (DA-ASAT)(cite definition).  If already in orbit
and maneuvered for intercept, it is known as a Co-Orbital
Antisatellite Missile (CO-ASAT)(cite definition).  At the moment, very
little evidence suggests that CO-ASAT systems are deployed.  Generally
speaking, co-orbital systems are largely employed for Rendezvous and
Proximity Operations (RPO) as will be discussed in greater detail
later.(cite brian's paper)

By process of elimination, that leaves DA-ASAT systems as the primary
threat, and as will be discussed later, they are indeed a threat.
DA-ASAT missiles behave much as one might expect.  In a torrent of
fire, they launch, and after reaching many thousands of miles per hour
collide with the target satellite turning the two into a puff of small
bits.(cite XXX) The kinetic energy of impact is sufficient to destroy
any target spacecraft (or any battleship for that matter given that
the relative collision velocity is likely to be circa than Mach 25 at
sea level).  In a manner similar to how popular films depict the
devastating effects of deploying tungsten rods or small asteroids to
impact population centers, the kinetic energy of a DA-ASAT missile
strike is likely to be devastating.  For context, spacecraft are
typically delicate enough that merely falling on their sides from a
height of 0m is sufficient to greatly damage them.  Like the furious
plasma of a nuclear blast, there is no known defense against a
successful collision.

\subsection{Debris Risk aka ``Kessler Syndrome''}

In 197whatever, while working with astronomical observations in the
asteroid belt, Donald Kessler wrote a paper which shifted the thinking
of space exploration near Earth: Collision Frequency of Artificial
Satellites: The Creation of a Debris Belt.  Known eponymously ever
since as ``Kessler Syndrome'', this paper outlines the forces of
statistics and physics that lead to the ``runaway'' creation of debris
conjuring scenes from popular films such as Gravity in which bits of
metal and other detritus swarm in clouds so thick that no spacecraft
dare tread.  While the reality is, much more mundane it does still
carry sufficient downside to cause concern.

Several misconceptions about Kessler Syndrome seem to exist, and a few
of them will be covered here for convenience.

We Need to ``Avoid'' Kessler Syndrome: While the worst of the effects
predicted by Kessler et al are certainly things we wish to avoid, it
is already too late to ``avoid'' Kessler Syndrome entirely.  Kessler et
al predict that once enough mass is in orbit, then the exponential
``runaway'' breakup will happen on its own (kessler-reunion).  As of
their 2016 paper, there is already enough mass in orbit to trigger
this scenario(kessler-reunion).  Putting that another way, we already
are in the early stages of exponential increase of debris, it just
doesn't ``feel'' like it because of the relatively small number of
collisions and active management.(XXX) To avoid the worst of the
effects, we already need, at minimum, to actively manage the
spacecraft and other detritus (such as second-stage rocket engines)
already in orbit.(kessler reunion)

Mega Constellations are a Mega Problem: This one is less of a
misconception, and more of a nuance.  We know from prior ASAT missile
test data that low altitude collisions don't last long in orbit, while
higher altitude collisions can last for centuries.(XXX) Most of the
planned mega constellations involve low altitude orbits where the risk
is greatly lessened.(XXX reference the appendix?)  Note that the
language chosen in this ``clarification'' is itself as clear as mud.
Quantifying the risk is a difficult proposition at best, but
qualifying is tractable.  Even if widespread collisions were to occur
in a mega constellation, the effects would likely be relatively short
compared with other historical collisions such as the Iridium/Kosmos
collision in 2009.(cite my own youtube video maybe?)

Kessler Syndrome will Entirely Deny Access to Space: FIXME: check with
brain on how to deal with this one - he has mentioned this one
specifically in the past.  Probably do a little proof here talking
about the volume of stuff in orbit vs the volume of orbit overall and
how small the debris would have to be ground in order for the mean
distance between pieces to be something like 1m.

ASAT Missiles will ``Cause'' Kessler Syndrome: As mentioned above,
Kessler Syndrome is already underway and is currently kept at bay by
active management and by virtue of still being in the early stages.
If Kessler Syndrome is thought of as an exponential growth of debris,
ASAT missiles effectively cause a ``time shift'' moving us further to
the right on that curve.  The question is further complicated when
considering the various critical factors such as the orbital altitude
of the collision, relative angle, relative velocity, etc.  The author
has, as yet, been unable to find any publicly available analyses
related to the amount of time-shift associated with any given launch.
However, it is worth noting that the more debris there is in orbit
already, the less it matters if we have one more launch.  Similar to a
pileup on a freeway during a snowstorm, the second collision may
double the total number of wrecked cars, but the 100th collision has
little impact on the overall picture.
