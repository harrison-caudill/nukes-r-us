\section{Overview of \acf{asat} Weapons}

The simplest meaning of an \acf{asat} weapon, is a weapon designed to
debilitate or destroy a satellite.  The concept of a ``weapon'' can
cover quite a bit of territory.  Perhaps the most common form of
\ac{asat} weapon is the \ac{dasat} which propels a \ac{kkv} toward a
target spacecraft.  Upon impact, the \ac{kkv} and the target generally
transform into a puff of problematic debris.  \acs{asat} Weapon and
\acs{asat} Missile are often used interchangeably.  However, a
distinction will need to be drawn between kinetic and non-kinetic
weapons.  More precisely, the distinction between those weapons that
\emph{do} generate debris and those that \emph{do not} needs to be
discussed.  Kinetic weapons which generate debris will be referred to
as \acp{kin}.  Those \ac{asat} weapons which do not generate debris
will be referred to as \acp{safe}.

\subsection{A Brief History of \ac{asat} Development}

This brief history of \ac{asat} weapons is included here for
convenience.  For a more thorough treatment of the topic, see Laura
Grego's {\it A History of Anti-Satellite Programs}\cite{grego} and
Brian Weeden's {\it Global Counterspace Capabilities, An Open Source
  Assessment}\cite{brian}.

Generation of orbital debris dates back to early explorations by the
Soviets in the late 1960's with the targeting of Kosmos
248.\cite[05-01]{brian}.  As Laura Grego states:
\begin{blockquote}
  Countries recognized that satellites would have great military value
  even before any had been successfully launched into
  orbit.\cite[2]{grego}
\end{blockquote}
which provides an obvious and powerful incentive to find a way to
neutralize an adversary's orbital assets. That incentive resulted in
early testing and some early successes, with \acp{coasat}.
\acs{coasat} interceptors are satellites placed in orbit, which can
maneuver to intercept a target.  Early Soviet designs were intended to
intercept a \ac{leo} target within 1-2 orbits (or 1.5 to 3 hours).
\acp{dasat} on the other hand, launched from the ground or an
airplane, were technologically more difficult\cite[4]{grego} and were
expected to take longer due to the necessity of waiting for a launch
window \cite[3]{grego}.

Because satellites operate under similar conditions as incoming
\acp{icbm}, \ac{abm} systems have a great deal of overlap with
\ac{asat} missile systems.  The net result is that the line between an
ABM system and an \ac{asat} system is all but indistinguishable.  As
time has progressed, most \ac{kkv} systems are \ac{abm} missiles.(XXX:
Verify this statement and find a good source, probably either grego or
brian).

Testing moratoria, \ac{abm} treaties, and changing economic and
political climates all played a role throughout the ensuing decades as
interest in \ac{asat} systems ebbed and
flowed.\cite[XXX]{grego} As space technologies advanced, so too
did the arms race between the great powers and ``...the United States
and the Soviet Union/ Russia have followed parallel and often mutually
reinforcing paths toward the militarization of space over the past 50
years.''\cite[2]{grego}

More recently, interest in \ac{coasat} systems has begun to resurface,
with close-approach imaging systems \cite[the russian one maybe?]{xxx}
and physical manipulation.\cite[chinese one maybe?]{xxx} Brian covers
some of this, as does Laura.)  While a transition away from
ground-based interceptors intended for kinetic kill is considered
beneficial, these weapons are still far from prolific enough to be
considered a viable alternative in the face of of imminent conflict.

\subsection{\acf{kin}}

Kinetic kill is arguably the oldest and most basic of kill mechanisms
to which the human race has access: you hit the target with a
rock.\cite[needed?]{xxx} There are a few key differences, as one might
imagine, between hitting something with a rock and employing a
multimillion dollar antisatellite missile.  The proverbial rock in an
\ac{asat} missile, usually referred to as the \acf{kkv}, contains some
amount of targeting and course-correction capabilities.\cite{sm3} In
addition to being a multimillion dollar smart rock, \acp{kkv} can
impact the target at many thousands of miles per hour (or at
hypervelocity) putting even the best baseball pitchers to
shame.\cite[kkv impact simulation]{xxx} Satellites, it should be
noted, do not generally fare well with even low-speed
collisions.\cite{whoopsies} Furthermore, even battleships do not fare
well against hunks of metal traveling at many thousands of miles per
hour \cite[any hypersonic paper]{xxx}.  At hypervelocity speeds, even
a fleck of paint can damage a window.\cite[iss window]{xxx} A
satellite can survive a hypervelocity impact with a \ac{kkv} just
about as well as one might imagine.\cite[hypervelocity impact
  simulation]{xxx} The primary objective of kinetic antisatellite
missiles, much the same as the primary objective of bullets, is to
ensure that something moving very very quickly physically collides
with the target.\cite[needed?]{xxx} This object could be the primary
\ac{kkv} itself, or, like the case of our ABM systems, a balloon
filled with tungsten rods.\cite[dig that one up or get rid of it]{xxx}

There is one notable exception to this description of \ac{asat}
missiles.  In the 1960's, the Soviet Union conducted a series of
experiments with what are probably best described as ``flying
bombs''.\cite[xxx]{brian} In 19XX, for example, the Soviet Union
maneuvered satellite XXX such that it closely approached satellite
YYY.\cite[brian or grego]{xxx} They then detonated XXX causing the
debris from that detonation then collided with satellite YYY ending
its mission with extreme prejudice.\cite[xxx pick something]{xxx}
Other than a few historical curiosities, the public records does not
contain significant evidence of this approach being further employed
or even planned.  The nearest modern approximation is likely to be a
refueling system.

There are two ways that a \ac{kkv} can approach a target: either from
the ground, or from orbit.\cite[overview of asat missile paper
  needed]{xxx} If a missile is launched from the ground, it is
typically known as a \acf{dasat}.\cite[definition from brian?]{xxx} If
already in orbit and maneuvered for intercept, it is known as a
\acf{coasat}.\cite[definition from brian?]{xxx} At the moment, very
little evidence suggests that \ac{coasat} systems are widely deployed.
Generally speaking, co-orbital systems are largely employed for
\acf{rpo} as will be discussed in greater detail
later.\cite[xxx]{brian}.

By process of elimination, that leaves \ac{dasat} systems as the
primary threat, and as will be discussed later, they are indeed a
serious threat.  \ac{dasat} missiles behave much as one might expect.
In a torrent of fire, they launch, and after reaching many thousands
of miles per hour collide with the target satellite turning the two
into a puff of small bits.\cite[kkv impact simulation]{xxx} The
kinetic energy of impact is sufficient to destroy any target
spacecraft (or any battleship for that matter given that the relative
collision velocity is likely to be circa than Mach 25 at sea
level).\footnote{A 100kg missile traveling at Mach 25 (at sea level on
a normal day) has approximately 1 kiloton of kinetic energy, which is
about 7\% of a Hiroshima bomb and well inside the working definition
of a ``tactical'' nuclear weapon. XXX: Put this computation into the
spreadsheet of doom to be released with the paper.} In a manner
similar to how popular films depict the devastating effects of
deploying tungsten rods or small asteroids to impact population
centers, the kinetic energy of a \ac{dasat} missile strike is likely
to be devastating.  For context, spacecraft are typically delicate
enough that merely falling on their sides from a height of 0m is
sufficient to greatly damage them.\cite{whoopsies} Like the furious
plasma of a nuclear blast, there is no known defense against a
successful collision.


\subsection{Debris Risk aka ``Kessler Syndrome''}

In 197whatever, while working with astronomical observations in the
asteroid belt, Donald Kessler wrote a paper which shifted the thinking
of space exploration near Earth: {\it Collision Frequency of
  Artificial Satellites: The Creation of a Debris
  Belt}.\cite{kessler-og} Known eponymously ever since as ``Kessler
Syndrome'', this paper outlines the forces of statistics and physics
that lead to the ``runaway'' creation of debris conjuring scenes from
popular films such as Gravity in which bits of metal and other
detritus swarm in clouds so thick that no spacecraft dare tread.
While the reality is much more mundane it does still carry sufficient
downside to cause serious concern.

Several misconceptions about Kessler Syndrome seem to exist, and a few
of them will be covered here for convenience.

\subsubsection{We Need to ``Avoid'' Kessler Syndrome}
While the worst of the effects predicted by Kessler et al are
certainly things we wish to avoid, it is already too late to ``avoid''
Kessler Syndrome entirely.  Kessler et al predict that once enough
mass is in orbit, then the exponential ``runaway'' breakup will happen
on its own.\cite[xxx]{kessler-reunion}.  As of their 2016 paper (XXX
link to paper?  list citation number?  title?), there is already
enough mass in orbit to trigger this
scenario.\cite[xxx]{kessler-reunion} Putting that another way, we
already are in the early stages of exponential increase of debris, it
just doesn't ``feel'' like it because of the relatively small number
of collisions and active management.(XXX: citation even necessary?
I'm stating my own interpretation of the cited paper's conclusions) To
avoid the worst of the effects, we already need, at minimum, to
actively manage the spacecraft and other detritus (such as
second-stage rocket engines) already in
orbit.\cite[xxx]{kessler-reunion}

\subsubsection{Mega Constellations are a Mega Problem}
This one is less of a misconception, and more of a nuance.  We know
from prior \ac{asat} missile test data that low altitude collisions
don't last long in orbit, while higher altitude collisions can last
for centuries.\cite{xxx} Most of the planned mega constellations
involve low altitude orbits where the risk is greatly
lessened.\cite[cite my own work maybe?]{xxx} Note that the language
chosen in this ``clarification'' is itself as clear as mud.
Quantifying the risk is a difficult proposition at best, but
qualifying is tractable.  Even if widespread collisions were to occur
in a mega constellation, the effects would likely be relatively short
compared with other historical collisions such as the Iridium/Kosmos
collision in 2009.\cite[my own youtube video maybe?]{xxx}

\subsubsection{Kessler Syndrome will Entirely Deny Access to Space}
FIXME: check with brain on how to deal with this one - he has
mentioned this one specifically in the past.  Probably do a little
proof here talking about the volume of stuff in orbit vs the volume of
orbit overall and how small the debris would have to be ground in
order for the mean distance between pieces to be something like 1m.

\subsubsection{\ac{asat} Missiles will ``Cause'' Kessler Syndrome}
As mentioned above, Kessler Syndrome is already underway and is
currently kept at bay by active management and by virtue of still
being in the early stages.  If Kessler Syndrome is thought of as an
exponential growth of debris, \ac{asat} missiles effectively cause a
``time shift'' moving us further to the right on that curve.  The
question is further complicated when considering the various critical
factors such as the orbital altitude of the collision, relative angle,
relative velocity, etc.  The author has, as yet, been unable to find
any publicly available analyses related to the amount of time-shift
associated with any given launch.  However, it is worth noting that
the more debris there is in orbit already, the less it matters if we
have one more launch.  Similar to a pileup on a freeway during a
snowstorm, the second collision may double the total number of wrecked
cars, but the 100th collision has little impact on the overall
picture.

\subsection{Previous Breakup Events}

Those not steeped in the mathematics and operational issues associated
with \ac{stm} may not fully appreciate the rancor associated with
events such as the 2007 destruction of Fengyun 1C in a Chinese
\ac{dasat} test.  Of all of the cases in recorded history where a
Breakup Event has occurred, in which two objects in Earth's orbit have
collided, a few stand out as singularly bad:

\begin{itemize}

\item The 2007 destruction of Fengyun 1C in a Chinese \ac{dasat} test.

\item The 2009 collision between Iridium 33 and Kosmos 2251

\item The 2021 destruction of Kosmos 1408 in a Russian \ac{dasat} ``test''.

\end{itemize}

Of the 4,379 pieces of tracked \ac{kin} debris that is still in orbit,
3,988 pieces are from those two \ac{dasat} missions representing
roughly 90\% of the tracked \ac{asat} debris.\cite[Table 5-1,
  p05-01]{brian} Because of the orbital altitude at which the Fengyun
1C destruction occurred, that debris is likely to be there for a very
long time.\cite{osa-debris} Roughly 1,500 pieces of debris remain in
orbit from the Iridium/Kosmos collision.

By definition, untracked debris is unknown.  While still dangerous,
many tens of thousands of pieces of untracked debris remain just so:
untracked.\cite[p05-01]{brian} Even a fleck of paint in orbit is
enough to damage windows on ISS.\cite[dig this one up]{xxx}.  If a
fleck of paint is sufficient to irreprabarably damage spacecraft
components, imagine what destruction could be unleashed by a
hypersonic collision resulting in a literal detonation of the
target.\cite[one of the hypersonic papers]{xxx}
