\section{Risk of Use in Combat}

Several analyses have been conducted, and they generally agree that
the use of KKVs in conflict is a credible threat.(cite XXX, YYY, ZZZ)
While the motivation for skinning cats is still somewhat unclear to
the author despite many decades of life, it does at least seem
somewhat settled and accepted that there are many ways to complete
that particular task.  Similarly, there are many ways to view a risk
assessment of ASAT systems.  That question is considered below with
particular emphasis on incentive structures.  This methodology is
employed as it will be shown to be instructive in the followup
question: ``What do we do about all of this?''

Two of the three main incentives listed below can be directly
mitigated in a self-reenforcing manner.  Collectively, these
mitigations constitute a tractable approach to reducing the potential
for harm from kinetic ASAT weapons.

\subsection{Underlying Premises}

There are three main premises, all related to incentives, outlined
here.  Further substantiation for and discussion of these premises can
be found in later sections.

ASATs are of Strategic Value: ASAT missiles are effective at
eliminating strategic targets like surveillance systems or even other
co-orbital ASAT systems.  Much like nuclear weapons on Earth, the
incentives to test and field ASAT weapons are strong.  As a corollary,
attempts to ban testing or induce disarmament will likely meet similar
levels of success.

Few Alternatives to ASAT Missiles Exist: Currently there are few
publicly-disclosed and/or deployed options for a permanently disabling
attack on a spacecraft that do not generate debris.  In the event of
armed conflict, any inclination to eliminate a strategic asset in
orbit is likely to be satisfiable in one and only one way: KKV.

CNC Safeguards \& Oversight: The command-and-control structures for
ASAT systems are not generally known.  While information about
safeguards is readily available (or at least well studied) for nuclear
weapons, analogous information does not exist for ASAT KKVs.  ABM
interceptors whose usage-doctrines of rapid-response to prevent an
incoming ICBM from reaching its target can also be used for ASAT
missions, adding to the problem of proliferation.  Combined with
disastrous activity in the past, it seems reasonable to assume that
there are few of the same CNC safeguards in place for ASAT systems
that are found in nuclear weapons.  The more prolific the weapons, and
the less controlled they are, the more likely they are to be used.

Conclusion: Credible Risk in Conflict

The question under consideration is:

In the event of terrestrial kinetic conflict, is there credible risk
of deployment of KKV missiles?

Like everything from economics to damped-driven harmonic oscillators,
some factors encourage and some factors discourage.  Examining the
above premises under that lens, they can be restated as follows:

ASAT missiles accomplish a useful task making them attractive military options

Few alternatives exist to fulfill that mission making them the only way to get the job done
There doesn't seem to be much to stop their use in combat

It then becomes somewhat straightforward to answer the original
question.  There is a well-established motivation to neutralize enemy
spacecraft in the event of terrestrial conflict.  In an armed conflict
between major powers, it seems reasonable to reach for a tool that
gets the job done; it seems reasonable to assume that those nations
may have some motivation to use KKVs to eliminate enemy assets.  It
would also be reasonable to reach for a different tool that could also
get the job done, but few such tools exist.  Finally, not much is
known that would stop these militaries from using those tools.

Conclusion: Yes.  The apparent incentive structure necessitates
treating the threat of deployment of KKVs in the event of terrestrial
kinetic war between major powers as credible.

There is some amount of subtlety in the word choices above, and some
amount of implied subtlety as well.  For example, it does not quantify
or even qualify the probability of deployment, or discuss the scale of
the conflict necessary.  Like most pronouncements of scientists, the
conclusions presented are nuanced, and hedged.

This conclusion depends quite heavily and quite explicitly upon the 3
prior premises.  For example, suppose that premise 3 is actually
false, and that the USA, Russia, and China actually all do have strong
safeguards in place against launching KKVs with greater earnestness
than for nuclear weapons.  Even in penning that statement, however,
the probability seems ludicrously low.  Combined with a longstanding
tradition of considering nuclear weapons to be real threats and a
stated assumption that they would be likely used by autocratic leaders
who are so armed rather than permit themselves to lose power(cite that
paper by that colonel), the assumption seems reasonable for now.
These premises will be discussed in greater detail below including
``sanity checks'' like the one above.

Technologies may yet be discovered that can mitigate premise 1 and
make ASAT missiles less effective (perhaps stealth satellites,
high-impulse maneuvering, EM pulses, ECM, or decoys), but this paper
assumes that premise 1 will remain largely in effect for some time.
The recommended mitigations below address points 2 and 3 by offering
alternative solutions during conflict and by providing push-back to
the use of KKVs.

\subsection{Consensus Opinion}

Few sources could be found that viewed this question in a similar
manner.  However, many other examinations of this topic have been
performed and have come to a similar conclusion: the risk of the world
utilizing KKVs is real.

FIXME: This should be a subsection giving a good overview of how other
analysts have viewed this question and whether or not they agree.  Not
a lot exists out there with a similar construction so we'll probably
focus on the conclusion

