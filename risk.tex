\section{Risk of Use in Combat}

Unused weapons are of little risk to anyone; all of the destructive
potential is avoided if \acp{kasat} go unused.  The novel analysis
below examines the incentive structure around these weapons,
culminating in the unfortunate conclusion that there is a very real
risk of use.

Three aspects of the incentive structure are discussed below:
Efficacy, Availability, and Governance.  While the general consensus
seems to be in agreement that these weapons do pose a very real risk,
none are currently known to examine the incentive structure in a
similar manner.  This examination is critical, as will be seen, in
finding actionable mitigations.


\subsection{Underlying Premises}

There are three main premises, all related to incentives, outlined
here.  Further substantiation for and discussion of these premises can
be found in later sections.

\subsubsection{Efficacy: ASATs are of Strategic Value}
\acp{kasat} are effective at eliminating strategic targets like
surveillance systems or even other \acp{coasat}.  Much like nuclear
weapons on Earth, the incentives to test and field \ac{asat} weapons
are strong.  As a corollary, attempts to ban testing or induce
disarmament will likely meet similar levels of success.

\subsubsection{Availability: Few Alternatives Exist}
Currently there are few publicly-disclosed and/or deployed options for
a permanently disabling attack on a spacecraft that do not generate
debris.  In the event of armed conflict, any inclination to eliminate
a strategic asset in orbit is likely to be satisfiable in one and only
one way: \acp{kasat}.

\subsubsection{Governance: Few \ac{cnc} Safeguards are in Place}
The command-and-control structures for ASAT systems are not generally
known.  While information about safeguards is readily available (or at
least well studied) for nuclear weapons, analogous information does
not exist for \acp{kasat}.  \ac{abm} interceptors whose
usage-doctrines of rapid-response to prevent an incoming \ac{icbm}
from reaching its target can also be used for ASAT missions, adding to
the problem of proliferation.  Combined with disastrous activity in
the past, it seems reasonable to assume that there are few of the same
\ac{cnc} safeguards in place for \ac{asat} systems that are found in
nuclear weapons.  The more prolific the weapons, and the less
controlled they are, the more likely they are to be used.

\subsection{Conclusion: Credible Risk in Conflict}

The specific question under consideration is:

\begin{blockquote}
  In the event of terrestrial kinetic conflict, is there credible risk
  of deployment of \acfp{kasat}?
\end{blockquote}

Answering that question requires little beyond the premises listed
above, which can be restated as:

\begin{enumerate}

\item {\bf Efficacy:} \acfp{kasat} accomplish a useful task making
  them attractive military options.

\item {\bf Availability:} Few alternatives exist to fulfill that
  mission making them the only way to get the job done.

\item {\bf Governance:} There doesn't seem to be much to stop their
  use in combat.

\end{enumerate}

There is a well-established motivation to neutralize enemy spacecraft
in the event of terrestrial conflict.  It seems reasonable for the
military to reach for a tool that gets the job done.  It would also be
reasonable to reach for a different tool that could also get the job
done, but few such alternatives exist.  Finally, not much is known
that would stop these militaries from using those tools.

Conclusion: Yes.  The apparent incentive structure necessitates
treating the threat of deployment of \acp{kasat} in the event of
terrestrial kinetic war between major powers as credible.

There is some amount of subtlety in the word choices above, and some
amount of implied subtlety as well.  For example, it does not quantify
or even qualify the probability of deployment, or discuss the scale of
the conflict necessary.  Like most pronouncements of scientists, the
conclusions presented are nuanced, and hedged.

This conclusion depends quite heavily and quite explicitly upon the 3
prior premises.  For example, suppose that premise 3 is actually
false, and that the USA, Russia, and China actually all do have strong
safeguards in place against launching \acp{kasat} with greater
earnestness than for nuclear weapons.  Even in penning that statement,
however, the probability seems ludicrously low.  Combined with a
longstanding tradition of considering nuclear weapons to be real
threats and a stated assumption that they would be likely used by
autocratic leaders who are so armed rather than permit themselves to
lose power\cite[that colonel]{xxx}, the assumption seems reasonable
for now.

\subsection{Consensus Opinion}

Few sources could be found that viewed this question in a similar
manner.  However, many other examinations of this topic have been
performed and have come to a similar conclusion: the risk of the world
utilizing \acp{kasat} is real.

FIXME: This should be a subsection giving a good overview of how other
analysts have viewed this question and whether or not they agree.  Not
a lot exists out there with a similar construction so we'll probably
focus on the conclusion

