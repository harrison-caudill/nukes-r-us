\section{Risk of Use in Combat}
\label{section::risk}

Unused weapons are of little risk to anyone; all of the destructive
potential is avoided if they lie dormant.  While availability of a
weapon is certainly an integral component to the risk of that weapon
being used, it is not the entire story.  To make a determination of
risk, the analysis below examines the incentive structure around these
weapons, culminating in the unfortunate conclusion that there is a
very real risk of use.  While the conclusion may be widely shared, no
other analyses are known to examine the incentive structure in a
similar manner.

Three aspects of the incentive structure are discussed below:
Efficacy, Availability, and Governance.  This focus on incentive
structures is crucial as it provides actionable targets for risk
reduction.


\subsection{Underlying Premises}

The core thesis of this publication relies upon three premises much as
a mathematical proof relies upon theorems and axioms.  The premises
upon which this paper depends are summarized below.  Each of the three
premises (Efficacy, Availability, and Governance) are related to the
incentive structures affecting the development and deployment of
\acp{kasat}.  Further substantiation for these premises can be found
in \S\ref{section::support}.

\subsubsection{Efficacy: ASATs are of Strategic Value}
\acp{kasat} are effective at eliminating strategic targets like
surveillance systems or even other \acp{coasat}.  Much like nuclear
weapons on Earth, the incentives to test and field \ac{asat} weapons
are strong.  As a corollary, attempts to ban testing or induce
disarmament will likely meet similar levels of success.

\subsubsection{Availability: Few Alternatives Exist}
Currently there are few publicly-disclosed and/or deployed options for
a permanently disabling attack on a spacecraft that do not generate
debris.  In the event of armed conflict, any inclination to eliminate
a strategic asset in orbit is likely to be satisfiable in one and only
one way: \acp{kasat}.

\subsubsection{Governance: No Reason Not To}
Internal \ac{cnc} safeguards and effective external pressures appear
to be all but absent in the case of \ac{asat} missiles, and are
certainly absent for \acp{kasat} in general.  While information about
safeguards is readily available (or at least well studied) for nuclear
weapons, analogous information does not exist for \acp{kasat}.  The
world may condemn \ac{asat} missile tests, but as illustrated by the
2021 Nudol test by Russia, that condemnation is of little apparent
value in practice.

\subsection{Conclusion: Credible Risk in Conflict}

The specific question under consideration is:

\begin{blockquote}
  In the event of terrestrial kinetic conflict between world powers,
  is there credible risk of deployment of \acfp{kasat}?
\end{blockquote}

The question above is specifically worded for several reasons.  For
example, an argument could be made that the United States is already
engaged in a cyber/economic war with China and Russia, but not yet
engaged in a ``kinetic war''.  Additionally, it could be argued that
the western world is engaged in a proxy war with Russia via Ukraine,
but again, not a kinetic war.  ``Credible risk'' is the chosen
language as there is always ``some risk'' and never any ``certainty''.

The first step in answering that question is to examine the incentives
that derive directly from the above premises:

\begin{enumerate}

\item {\bf Efficacy:} There is a well-established motivation to
  neutralize enemy spacecraft in the event of terrestrial conflict.
  \acfp{kasat} accomplish a useful task making them attractive
  military options.

\item {\bf Availability:} Few alternatives exist to fulfill that
  mission making them the primary (if not exclusive) way to get the
  job done.

\item {\bf Governance:} There doesn't seem to be much to stop their
  use in combat.

\end{enumerate}

\begin{blockquote}
  {\bf Conclusion: Yes.  The apparent incentive structure necessitates
    treating the deployment of \acp{kasat} as a credible risk in the
    event of terrestrial kinetic conflict between world powers.}
\end{blockquote}

There is some amount of subtlety in the word choices above, and some
amount of implied subtlety as well.  For example, it does not quantify
or even qualify the probability of deployment, or discuss the scale of
the conflict necessary.


\subsection{Consensus Opinion}

Few sources could be found that viewed this question in a similar
manner.  However, many other examinations of this topic have been
performed and have come to a similar conclusion: the risk of the world
utilizing \acp{kasat} is real.  For example:

\begin{itemize}

\item Dr. David Write of the The Union of Concerned Scientists seem to
  agree that incentives matter stating that ``If such ASATs are seen
  as legitimate weapons, a country might therefore have a strong
  incentive to develop one for use against satellites that are deemed
  highly important militarily.''\cite[p2]{uocs-no-likey} However, he
  stops short of examining them in depth and instead recommends
  ``[prohibiting] the testing or use of destructive ASAT
  weapons''\cite[p2]{uocs-no-likey}.

\item The Carnegie Endowment for International Peace adopts a similar
  stance by discussing the perils of \acp{kasat} and calling for them
  to be banned, but without examining the magnitude or cause of the
  risk itself.\cite{carnegie-no-likey}

\item The proposal by the EU Non-Proliferation does discuss incentives
  to voluntarily cease all \ac{kasat} testing, but does not address
  the national incentives to test.\cite{eu-ban}

\end{itemize}

Most discussions of \acp{kasat} focus upon \ac{asat} missiles Few
analyses of the cause of that risk, however, seem to exist as
discussion generally begin with the premise that they are a risk
because they are a threat.  The distinction being that a threat is a
mechanism which can cause harm, whereas a risk is a probability of
that threat causing the aforementioned harm.  The threat of widespread
use of \acp{kasat} is sufficiently large, that most discussions focus
on that threat without consideration of the risk.

