\section{Supporting Premises}

\subsection{ASATs are of Strategic Value}

The characterization of kinetic ASATs as ``strategic'' weapons is
perhaps best begun with a comparison to nuclear armaments.  A target
is totally defenseless against a successful strike from either and
both produce debris that can last for millenia.  The world holds many
hard-earned fears of nuclear weapons that stretch well beyond the
initial wave of casualties.


Space debris can linger for millenia, denying access to that region of
orbital parameters There is no defense against a successful strike
Fallout can linger for millenia, denying access to that region There
is no defense against a successful strike

Satellites are strategic assets with capabilities ranging from
predicting the weather for air-force operations, to providing
Position, Navigation, and Timing (PNT), to monitoring for ICBM
launches.  As Laura Grego states, ``Remote-sensing satellites that
take high-resolution images of the ground have strategic and tactical
importance that makes them attractive targets for ASAT
weapons.''(grego, 16) Even something as simple as GPS provided a major
strategic advantage during the first Gulf War.(cite...sh** where was
that?? The one talking about iraqi's getting lost) Civilian
communications were made possible in Ukraine with Starlink satellites,
and communications with western advisors via Air Force Satellite
Communications (AF-SATCOM) (cite XXX).

Much as the ability to eliminate a strategic asset, such as a city,
makes an ICBM a strategic asset, so too does the ability to destroy a
$\$$500m satellite make a KKV a strategic asset.  Additionally,
satellites have little or no defense against these weapons, their
destruction is complete, and remediation requires the construction and
launch of a new satellite.  It seems reasonable to categorize them as
``strategic''.

Even if the precise distinction of classifying them as ``strategic''
vs ``tactical'' were to be disregarded there is one statement that can
clearly be made: they accomplish the mission.  If the mission is to
ensure that an enemy spacecraft is no longer available to that enemy,
then ASAT missiles will get the job done.

It should be noted that KKVs have never been used in time of war, and
so their odds of intercept could be reasonably questioned.  However,
as discussed above, the countermeasures that would likely be required
for a satellite in orbit to evade an incoming KKV can be quite severe
and are unlikely to have been deployed.(FIXME: Is this something I can
cite?)

See Brian Weeden's paper XXX, section YYY for further details
regarding the efficacy and distribution of these weapons.

\subsection{Few Alternatives to ASAT Missiles}

Some other weapons may yet be of even higher overall efficacy (such as
High Altitude EMP detonations), but the ability to eliminate a
nation's space infrastructure is undeniably of great strategic, if
potentially short-sighted, value.

Proving a negative, i.e. proving that the titular statement is true,
is obviously difficult.  For the sake of expediency, Brian Weeden's
excellent publication Counterspace XXX, will be treated as
authoritative and comprehensive - if it isn't in his publication, it
is assumed not to exist in the public record.

There are three main categories of counterspace weaponry:
\begin{enumerate}
\item Kinetic kill (e.g DA-ASAT missiles)
\item Electromagnetic (e.g. laser dazzling)
\item Cyber attacks (e.g. Viasat)
\end{enumerate}

There is a subtlety worthy of mention at play here: the concept of
entirely eliminating a strategic capability vs a temporary (or even
regional) denial of that capability.  LASER dazzling, for example, can
deny optical ISR capabilities whenever the satellite is within range
of the LASER, but it will continue to function otherwise.  RF jamming
is also, generally, only effective while it is active.  While it is
also possible to physically damage some ISR systems, the predominant
mode of operation is temporary denial of service.  Cyber attacks are
also generally viewed as temporary disruptions with the caveat that
the vulnerabilities can then be fixed.  There is also some precedent
for physical damage being brought about by a cyber attack (e.g. ROSAT,
Stuxnet, Viasat).

LASER dazzling of sufficient power may damage an optical sensor, and
cyber attacks may occasionally succeed at damaging a spacecraft.  If
employed under surprise conditions, these weapons can be devastating.
However, defenses do exist against these weapons.  Much like nuclear
weapons act terrestrially, there is little defense against a
successful strike from a KKV.

Some other capabilities are being, slowly, developed.  For example,
the recent announcement of putting billboards in orbit could provide
the necessary capabilities to perform RPO-based attacks.  The Chinese
have announced their ability and intent to utilize short-distance
grappling.  These sorts of attacks, which are capable of permanently
disabling a spacecraft without generating debris, are where weapons
development efforts should be focused.  Otherwise, the kinetic ASAT
missile will continue to be the go-to weapon in orbit.

\subsection{CNC Safeguards \& Oversight}

Note from the author: I do not consider myself to be enough of an
expert in these matters to do justice to these topics.  I also do not
have access to those whom I would consider to be experts.  I'll do my
best.  As with all of my publications, I not only welcome, but
actively encourage feedback from the community.  I can't fix a problem
I don't even know about.

While the chain of command, safeguards, and civilian oversight is well
studied (if not always well understood) for the world's nuclear
arsenals, little analogous information is available for ASAT missiles.
Furthermore, there are few public calls for change in this regard.
Without existing oversight and without even the necessary pressure to
add oversight, the kind of safeguards expected of nuclear weapons are
unlikely to be present for ASAT missiles.

\subsubsection{Anti Ballistic Missile Systems}
The United States primarily uses the RIM 161 ABM missile for
antisatellite missions.  ICBMs from Russia require approximately 30
minutes to reach their targets in the United States, and
submarine-fired missiles require only 15 minutes.(cnc-primer) As such,
much of US nuclear doctrine is built around the concept of rapid
response.(cnc-primer) Since the ABM shield is a part of that doctrine
and the US uses ABM missiles for antisatellite missions(grego has a
good quote about this), there is a natural incentive to be able to
fire quickly and easily.

No specific conclusions are drawn from this particular observation,
though a natural suspicion may develop that this rapid-response
doctrine (which seems entirely appropriate for protecting against
ICBMs) may complicate the process of adding CNC safeguards to KKVs.

\subsubsection{Civilian ``Oversight''}
Highly unpopular draft in Russia
Uighurs in China
Navalny
Social manipulation in the US

It has been argued that human power structures are governed by the
mathematics of their electorate.\cite[dictator]{fixme} In the case of
a democratic election in the United States, for example, that might
mean fabricating rationalizations to provide incentives to critical
blocks of supporters.  If too many promises are broken, in theory, the
electorate might facilitate a transition of power.  In the case of
autocratic nations, like China and Russia, the concept of civilian
oversight seems a bit less powerful.  When non-conforming citizens are
enslaved (Uighurs), when dissenting politicians are imprisoned
(Navalny), it seems reasonable to assume that the general civilian
population.  As can be clearly seen by China's 2007 launch and by
Russia's 2021 launch, the responsible scientifically minded citizens
of those nations had little control.  The Dictator's Handbook provides
an excellent discussion on the merits of electorate size.


\subsubsection{United States}
``The U.S. President has sole authority to authorize the use of
U.S. nuclear weapons.''  The destructive power and global consequences
of an unauthorized launch is so well respected that ``[Nuclear
  Command, Control, and Communications] requires rigorous procedures
and processes to support the President and the Secretary of Defense in
exercising command authorities in the areas of situation monitoring,
decision-making, force direction, force management, and planning to
direct the actions of the people who operate nuclear
systems.''(nuke-matters-handbook)

Several sections of US Code and at least two recent bills relate to
the permissible use of nuclear weapons.  Title 10 of US Code, for
example, pertains to the armed forces at large and establishes a
Nuclear Weapons Council.  Title 50 governs war and national defense
and contains several chapters related to nuclear weapons.  These
titles, by contrast, do not contain guidance, restrictions, or
civilian oversight for antisatellite weapons with a few minor
exceptions.

The US Navy, US Strategic Command, US Space Force, and US Space
Command were all contacted in advance of publication.  All declined to
comment, as did their officers.  If there is a use-of-force doctrine,
CNC safeguards, or any other governance structure in place, they are
unwilling to share it. The CNC safeguards for American KKVs remains
unknown.

\subsubsection{China}

There is, at best, contradictory evidence regarding the stability of
the link between the CCP and the PLA.  Though, much of that evidence
seems to predate Xi.  It also sounds like the 2007 test may have taken
place without knowledge/consent of the CCP. - Found a good paper from
the congressional research service and will stick with its
conclusions.  The fact that China executed the 2007 test suggests
that, at the very least in 2007, civilian scientists with knowledge
and understanding of the consequences of that kind of launch were not
in a position to provide substantive pushback.  The author is not
aware of any evidence to suggest that has changed.  Whether or not
that has changed only underscores the fact that little public
information about the CNC safeguards that may or may not be in place
in China is available.

Previous ASAT testing was performed under the command of general XXX,
YYY of the PLA Air Force, a position currently held by ZZZ.  ZZZ's
history does not seem to include orbital mechanics, quantum physics
(radiation pressure), solar science (changes in atmospheric drag), or
space systems in general.  All of this uncertainty only underscores
the concerns that exist around the fact that China is one of the two
truly irresponsible global actors in the field of ASAT systems.

\subsubsection{Russia}

Generally regarded as an oligarchy, Russia's true electorate are the
billionaire oligarchs.

FIXME: I need help here.  Much of the effective leadership of Russia
is steeped in cold-war traditions from the USSR.(XXX: is this even
true????  Putin certainly is, but what about his generals??)  In spite
of the lessons learned

They have performed many tests, even one quite recently despite all of the lessons learned from prior tests by the russians themselves even.
That they destroyed a satellite at such a high orbital altitude suggests, much like the 2007 chinese test, civilian oversight was not part of the process beyond a likely directive from putin - I have NO substantiation for half of this sh**...I need a russian expert to help here... .or at least a good source...
FIXME: Anything about CNC for russian nukes?

\subsubsection{India}
India's first and only ASAT missile test (Mission Shakti) occurred at
250km of orbital altitude, which clearly indicates that appropriate
guidance did at least reach the appropriate governing entities.

FIXME: What to put here?

