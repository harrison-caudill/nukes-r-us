\section{Supporting Premises}
\label{section::support}

If the three premises regarding the incentive structure surrounding
\acp{kasat} are accepted, this section may be safely reserved as a
reference.

\subsection{ASATs are of Strategic Value}

Satellites are strategic assets with capabilities ranging from
predicting the weather for air-force operations, to providing
\ac{pnt}, to monitoring for \ac{icbm} launches.  As Laura Grego
states, ``Remote-sensing satellites that take high-resolution images
of the ground have strategic and tactical importance that makes them
attractive targets for ASAT weapons.''\cite[p16]{grego} Even something
as simple as \ac{gps} provided a major strategic advantage during the
first Gulf War.\cite{gps-gulf}

Much as the ability to eliminate a strategic location/installation
makes an \ac{icbm} a strategic asset, so too does the ability to
destroy a billion dollar satellite make a \ac{kasat} a strategic
weapon.  Satellites have little or no defense against these weapons,
their destruction is complete, and remediation requires the
construction and launch of a new satellite.  It seems reasonable to
categorize them as ``strategic''.  There is an argument that
\acp{dasat} are already viewed as strategic
deterrents.\cite{war-no-more}

Even if the precise distinction of classifying them as ``strategic''
vs ``tactical'' were to be disregarded there is one statement that can
clearly be made: they accomplish the mission.  If the mission is to
ensure that an enemy spacecraft is no longer available to that enemy,
then \acp{kasat} will get the job done.  Furthermore, it will quickly
and definitively be known whether or not any \ac{kasat} attack was
successful as any successful attack will result in the generation of
observable debris.

It should be noted that \acp{kasat} have never been used in time of
war\cite{brian}, and so their odds of intercept could be reasonably
questioned.  However, as discussed above, the countermeasures that
would likely be required for a satellite in orbit to evade an incoming
\ac{kkv} (see \S \ref{section::mitigation::1}) can be quite severe and
seem unlikely to have been deployed.

\subsection{Few Alternatives to \acp{kasat} Exist}

To pose a viable alternative to a \ac{kasat}, it must have two key
attributes: its effects must be permanent and also complete.  LASER
dazzling, for example, is only effective while the target is within
range of the LASER and is only effective against an optical payload;
LASER dazzling is neither permanent nor complete.  If the LASER were
powerful enough to physically damage the optical sensor, it could be
said to be ``permanent'', but without disabling the other critical
systems, it cannot be said to be ``complete''.

Common \ac{asat} methods are listed in figure
\ref{figure::asatOptions} with these attributes annotated.  There are
some caveats associated with these findings\footnote{See the Executive
Summary from
\href{https://swfound.org/media/207344/swf_global_counterspace_capabilities_2022.pdf}{Global
  Counterspace Capabilities} (source \cite{brian}) for a more thorough
examination of the state of public evidence related to \acl{asat}
weapons.}, for example ``While the United States does not have an
operational, acknowledged direct ascent anti-satellite (DA-ASAT)
capability, it does have operational midcourse missile defense
interceptors that have been demonstrated in an ASAT role against a low
LEO satellite.''\cite[pxiii]{brian} It is also unclear whether or not
Russia's \ac{dasat} system (Nudol) is or will be
deployed.\cite[pxv]{brian}

For the sake of expediency, Brian Weeden and Victoria Samson's
publication {\it Global Counterspace Capabilities}\cite{brian} will be
treated as authoritative and comprehensive - if it isn't in that
publication, it is assumed not to exist in the public record.

\begin{figure}
  \centering
  \begin{tblr}[
    ]{%
      colspec = {
        Q[c,m]Q[c,m]Q[c,m]|
        Q[c,m]Q[c,m]Q[c,m]Q[c,m]
      },
      column{2-7} = {font=\bfseries},
      row{even} = {gray!25},
      row{odd} = {white},
      row{1-2} = {gray!65},
    }

    \SetRow{rowsep=10pt}
    \SetCell[c=3]{c,m} {\bf Weapon} & & &
    \SetCell[c=4]{c,m} {\bf Deployable?} \\

    \SetRow{c,b}
    {\bf Weapon}
    & \begin{turn}{90} {\bf Permanent?} \end{turn}
    & \begin{turn}{90} {\bf Complete?} \end{turn}
    & \begin{turn}{90} {\bf USA} \end{turn}
    & \begin{turn}{90} {\bf Russia} \end{turn}
    & \begin{turn}{90} {\bf China} \end{turn}
    & \begin{turn}{90} {\bf India} \end{turn}
    \\

    \ac{leo} \acs{dasat} & \CIRCLE & \CIRCLE
    & \CIRCLE
    & \CIRCLE
    & \CIRCLE
    & \LEFTcircle
    \\

    \ac{leo} \acs{coasat} & \CIRCLE & \CIRCLE
    & \Circle
    & \CIRCLE
    & \Circle
    & \Circle
    \\

    Directed Energy & \LEFTcircle & \Circle
    & \CIRCLE
    & \CIRCLE
    & \Circle
    & \Circle
    \\

    Cyber \ac{leo} & \LEFTcircle & \LEFTcircle
    & \CIRCLE
    & \CIRCLE
    & \CIRCLE
    & \CIRCLE
    \\

  \end{tblr}
  \caption{Availability and capabilities of various \acl{asat} weapons
    systems.}

  \label{figure::asatOptions}

\end{figure}

Speculation exists regarding the development of other \ac{asat}
methods.  For example, the hardware, software, and operational
capabilities necessary for the recently hypothesized billboard
constellation\cite{billy-boy} could potentially also be used to
execute \ac{rpo} based attacks.  The Chinese have announced their
ability and intent to utilize short-distance
grappling.\cite[p03-04]{brian} These sorts of attacks, which are
capable of permanently and completely disabling a spacecraft without
generating debris, are where weapons development efforts should be
focused.  Otherwise, the \ac{kasat} will continue to be the go-to
weapon in orbit.


\subsection{Internal \& External Governance}

No documentation was found that directly illustrates military
doctrine, civilian oversight mechanisms, chains of command,
use-of-force doctrines, etc related to \acp{kasat} for any nation.
Several branches of the US military were contacted in advance of this
publication seeking information on any of these topics: all declined
to comment.  Worse-yet, the concept of civilian oversight is also
somewhat muddy in the more autocratic nations.  Below is a summary of
the examined sources, and resulting findings.


\subsubsection{\acfp{abms}}

The primary mission of an \acf{abms} is to intercept an incoming
\ac{icbm}.  With development of these systems stretching back to the
1950's\cite[p01-10]{brian}, sophisticated \acp{abms} have now been
deployed.\cite[p01-15]{brian} \acp{icbm} from Russia require
approximately 30 minutes to reach their targets in the United States,
and submarine-fired missiles require only 15 minutes.\cite{cnc-primer}
As such, much of US nuclear doctrine is built around the concept of
rapid response.\cite{cnc-primer}

Incoming \acp{icbm} operate in a similar regime to orbiting
satellites, traveling at similar speeds and
altitudes.\cite[p01-15]{brian} As such, \acp{abms} can and have been
used for \ac{kasat} missions.  With a ``one-time software
modification'', three American SM-3 missiles were known to have been
made capable of intercepting satellites.\cite[p01-15]{brian}

Ultimately, missiles from an \ac{abms} are capable of executing
\ac{kasat} missions and are under a governance structure that likely
(owing to the nature of their mission) favors rapid action rather than
prudent restraint.  The number of missiles that have been modified to
accomplish that mission by the US alone is unknowable, and no public
material is available discussing any \ac{cnc} safeguards or aspects of
the chain of command related to \ac{kasat} missions.

\subsubsection{Civilian ``Oversight''}

It has been argued that human power structures are governed by the
ability to provide rewards to key supporters.\cite[ch1]{dictator} In
small coalition systems, institutions are fragile and readily
overturned, especially during a change of power.\cite[ch2]{dictator}
In small coalition governments, such as China, the populace at large
(and their interests) have little sway over
decision-making.\cite[ch1]{dictator}

In China, citizens are literally enslaved due to their professed
religion.\cite{uyghurs} It is not uncommon for political rivals to be
assassinated\cite{polonium} or imprisoned\cite{navalny} in Russia and
those in the general population who object are regularly
jailed.\cite{protestors-jailed} While some say that democracy in a
large-coalition government like the United States is
stable\cite[ch10]{dictator}, it can be rather unnerving to consider
that the nation only narrowly evaded what could reasonably be
described as an attempted coup
d'etat.\cite{coup-coup-clock}\cite{coup-coup-click}

What impact, if any, that ``civilian oversight'' might have on
\ac{kasat} governance in a small-coalition government like China or
Russia (or the United States should ensuing coup-attempts such as the
proposed constitutional convention succeed) is beyond the scope of
this paper.  This concept in this context is an open area of research
(see \S\ref{section::furtherWork}).

See Bruce Bueno de Mesquita and Alastair Smith's book
\href{https://www.publicaffairsbooks.com/titles/bruce-bueno-de-mesquita/the-dictators-handbook/9781541701366/}{The
  Dictator's Handbook} for a more thorough treatment of
coalition-based governance.

\subsection{International Response}

In November, 2021, Russia launched a Nudol missile and destroyed
Kosmos XXX.\cite[nudol]{xxx} This launch was met with widespread
condemnation.\cite[nytimes nudol response]{xxx} It seems reasonable to
assume that Putin would have known that widespread condemnation would
follow, and yet the launch happened anyway.  No significant economic
or military repercussions resulted from that action.  It would appear
as though international condemnation is not much of a deterrent.

\subsubsection{United States}

The United States has sophisticated \ac{cnc} safeguards in place, as
well as published doctrine and sever sections of civilian code
associated with the management and use of nuclear weapons.\cite[the
  right joint publication]{xxx} Nuclear weapons possess terrible
destructive power and are a central element of the \ac{mad}
doctrine.\cite{getting-mad} The destructive power and global
consequences of an unauthorized launch is so well respected that
``[Nuclear Command, Control, and Communications] requires rigorous
procedures and processes to support the President and the Secretary of
Defense in exercising command authorities in the areas of situation
monitoring, decision-making, force direction, force management, and
planning to direct the actions of the people who operate nuclear
systems.''\cite{nuke-matters-handbook} Published materials (\ac{cnc}
safeguards, published doctrine, etc) for \acp{kasat}, by contrast, do
not appear in obvious and/or analogous locations.

Several sections of US Code and at least two recent
bills\cite[bill-1]{xxx}\cite[bill-2]{xxx} relate to the permissible
use of nuclear weapons.  Title 10 of US Code, for example, pertains to
the armed forces at large and establishes a Nuclear Weapons Council.
Title 50 governs war and national defense and contains several
chapters related to nuclear weapons.  These titles, by contrast, do
not contain guidance, restrictions, or civilian oversight for
antisatellite weapons with a few minor exceptions.  While the
structural similarities of appropriate governance would not necessarily
be expected carry over to locations in law, one could reasonably
expect that were a governance structure in place at all, it might
appear somewhere in the \acf{usc}.

\fixme{Finish the spreadsheet for jcs.mil and put that result in
  here.}

The US Navy, US Strategic Command, US Missile Defense Agency, US Space
Force, and US Space Command were all contacted in advance of
publication seeking comment on their \ac{cnc} safeguards and/or
doctrine regarding \ac{asat} missions.  All declined to comment.  If
there is a use-of-force doctrine, CNC safeguards, or any other
governance structure in place, they are unwilling to share it with the
author.

While the United States does have the most publicly-available
information of any of the nations with current \ac{kasat}
capabilities, the governance structure for American \acp{kasat}
remains unknown.


\subsubsection{China}

The 2007 destruction of Fengyun-1C was arguably the most irresponsible
space weapons test ever to have happened.  It produced an initial wave
of roughly 3,500 pieces of trackable debris $\ge$10cm, of which
approximately 80\% (2,750 pieces) are still in
orbit.\cite[p05-01]{brian} The orbital altitude of the test (865km)
results in debris with an expected lifetime measured in hundreds of
years.\cite[fig from SpaceX]{rando-orbit} The mere fact that the 2007
test occurred suggests that, at the very least in 2007, civilian
scientists with knowledge and understanding of the consequences of
that kind of launch were not in a position to provide substantive
pushback.

In their 2015 defense whitepaper\cite{china-debutant}, China
officially designated space as a warfighting domain, a move that NATO
would not make until 2019.\cite{nato-likes-space} China's space
warfare is managed by the \ac{ssd} under the
\ac{ssf}.\cite[p65]{osd-china-21} At present, ``General Li Fengbiao is
the SSF commander. Lt. Gen. Shang Hong and Lt. Gen Ju Qiansheng are
the commanders of the Space Systems and Network Systems Departments,
respectively, though new promotions were announced in 2021.''
\cite[p65]{osd-china-21} Official biographies for these individuals do
not appear to exist, and a cursory examination of other public sources
does not indicate any specific expertise in the areas necessary to
truly comprehend the consequences of \acp{kasat}.

All of this uncertainty only underscores the concerns that exist
around the fact that China is one of the two truly irresponsible
global actors in the field of \acp{kasat} and lacks any known internal
governance to prevent further abuses.


\subsubsection{Russia}

Russia is a small-coalition oligarchy.\cite[russian-gov-primer]{xxx}
Political adversaries are routinely imprisoned and/or
assassinated.\cite{navalny}\cite{polonium} Threats of nuclear
launches, credible allegations of plots to launch a false-flag attack
with a dirty bomb, and a road to economic ruin are all a part of the
current ruling coalition's actions related just to the Ukranian
invasion of
2022.\cite{false-flag}\cite{russian-economy}\cite{putin-compensating}

How this regime would view use of the Nudol antisatellite missile
should the war in Ukraine escalate is well beyond the scope of this
paper.

\subsubsection{India}
India's first and only \ac{kasat} test (Mission Shakti) occurred at
220km of orbital altitude leaving very little debris which quickly
decayed.\cite[p05-01]{brian} By comparison, India has been a beacon of
restraint and good judgment.  However, a single instance of good
judgment is not a guarantee of future behavior.  The lack of
published and verifiable material makes for a very uncertain future.
