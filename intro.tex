\phantomsection
\addcontentsline{toc}{section}{Introduction}

\titlenote{}

\maketitle

\pagestyle{theRest}
\thispagestyle{firstPage}

Since it was first discovered that a large stick could be employed by
one primate to beat another primate to death, the application of
kinetic energy in war has been a mainstay.  In the terrestrial domain,
that means bazookas and bullets, missiles and mines; in the orbital
domain, that means \acp{kasat}.  While we do have some fundamentally
new techniques such as cyber warfare\cite{big-risks}, and while the
precision and quantity of application has changed, most approaches in
warfare, even in orbit, come down to application of kinetic
energy.\cite{brian} We now have sniper bullets that can punch through
plate steel then turn into a grenade\cite{fancy-bullets}, and
hypersonic missiles that can circle the globe.\cite{china-is-hyper}
Though the approach of pumping kinetic energy into a target via a
bullet, or bomb, or pressure-wave may work wonderfully in the
terrestrial domain, it produces a rather unfortunate hazard when
employed in orbit: space debris.

The term ``space debris'' is quite encompassing.  It can refer to old
second-stage rocket engines or it can refer to a fleck of
paint\cite{paint-is-power}.  Satellites in \ac{leo} travel at around
30,000 kmh (18,000 mph or 8,000 m/s). At those speeds, it doesn't take
much to inflict physical damage.  Thousands upon thousands of pieces
of debris have been meticulously observed and
cataloged\cite{debris-101}, with untold thousands more out there that
we simply cannot observe; after all observing a fleck of paint from a
1960s rocket engine traveling at Mach 23 in the dark is a rather
difficult task.  This orbiting and largely unobserved debris poses
serious risks to space travel.  The fundamental principle on which the
vast majority of weapons-development is based is fundamentally
incompatible with humanity's future in space just like use of nuclear
weapons is fundamentally incompatible with humanity's future on Earth.
We cannot allow our most effective counterspace weapons, \acp{kasat},
to be deployed just like we cannot allow our most effective bombs,
fusion bombs, to be deployed.

Many parallels exist between nuclear armaments applied terrestrially
and \acp{kasat} applied in orbit.  Both are wildly destructive, both
are strategic weapons, both are major environmental hazards, both
would be highly effective in combat, and neither is going away
entirely in the foreseeable future.  Most attempts to mitigate the
destructive effects of \acp{kasat} tend to focus around the concept of
banning them.\cite{early-to-call} If the risk of widespread use of
\acp{kasat} is high, then banning them is unlikely to happen for the
same reasons that nuclear weapons will likely never be banned.  If the
risk of deployment is low, then banning them is a marginal value-add,
but of little real consequence.  Either way, attempting to ban
\acp{kasat} seems unlikely to have any measurable effect on the risk.

By examining the incentive structures associated with \acp{kasat},
this paper finds actionable mitigations that hold true promise of
decreasing the risk to humanity's future in space.  Using lessons
learned from attempts at nuclear governance and pollution
mitigation/remediation, we can begin to find steps that can be taken
that can productively move us away from the perilous cliff on which we
now stand.
