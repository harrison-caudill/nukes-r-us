\section{Mitigation}

Evaluation of the risk posed by \acp{kasat} was performed in the
context of the corresponding incentives because it provides clear
targets for mitigation.  By altering the incentives to
build/test/deploy these terrible weapons, and by further adding
disincentives, one might reasonably attenuate their odds of use.  The
mitigation strategies below are tied directly to the incentive
structure outlined above in \S\ref{section::risk} and further
substantiated/developed in \S\ref{section::support}.

\subsection{Mitigation of Premise 1: Divide and Hide}
\label{section::mitigation::1}

While there is reason to believe that there are clandestine efforts
designed to limit the efficacy of \acp{kkv} against individual assets,
there is little public evidence of these efforts, and even less about
wide-scale deployment.  Furthermore, the hypothesized methods below
would likely be cost prohibitive.  It might be more effective to
conduct other attacks on the infrastructure employed for \acf{ssa}.

Some amount of benefit can be derived by transitioning away from large
singular assets and towards resilient constellations.  However,
disaggregation cannot change the fundamental utility of kinetic
attacks in general.  The same forces of economics and technological
advancement that make constellations feasible would likely also serve
to decrease cost and increase availability of \acp{kasat}.  Smaller
and less expensive launch vehicles, for example, have capabilities
that are similar to \acp{dasat}.

As a thought experiment, consider the following approaches to decrease
the efficacy of a given \ac{kasat}:

\textbf{Stealth Satellites:} Decreasing RADAR cross-sections and IR
signatures might make targeting satellites more difficult and improve
survivability in the event of an attack.  As the Union of Concerned
Scientists found, a surface temperature close to the boiling point of
liquid Nitrogen would likely be appropriate.\cite[p48]{abm-me-not}

\textbf{High Impulse Thrusters:} In principle, it might be feasible to
incorporate high-impulse thrusters to perform an evasive maneuver.
When coupled with high-accuracy \ac{ssa}, it could conceivably permit
a spacecraft to evade an attack.

\textbf{Kinetic Counteroffensive:} In 1974, the Soviet Union launched
and demonstrated a machine gun on the Salyut-3
spacecraft.\cite[p02-04]{brian} A relatively simple system such as
Metal Storm might be capable of initiating an early breakup of the
\ac{kkv}.  This approach has some obvious difficulties beyond basic
execution such as the generation of a large debris field still on
track to intersect with its target.

\textbf{\acf{ecm}:} Utilizing newer wide bandgap semiconductor
systems, it might be possible to deliver an impulse with sufficient
field strength to electronically disable the \acpos{kkv} targeting and
maneuvering systems.  Obvious difficulties include things like
targeting, and the ability to conduct the \ac{ecm} attack while there
is still time to then evade it.  This technology is currently being
fielded terrestrially by companies such as Epirus.\cite{epirus}

The obvious and straightforward methods one might conjure to avoid
collision with a modern \ac{kkv} are, to put it charitably,
problematic.  Lacking any obviously viable path to mitigate premise 1,
and lacking any public evidence of these efforts, it is assumed that
Premise 1 will remain in effect for the foreseeable future.  If
\acp{kasat} continue to be effective, then there will be continued
incentive to use them.

\subsection{Mitigation of Premise 2: Arming Adversaries}
This mitigation is, perhaps, the most disturbing of the three: the
world would be a better place if our adversaries had more weapons.
More specifically, the risk of accelerating Kessler Syndrome would be
lessened if all nations with existing \ac{kasat} capabilities (most
especially Russia and China) were in possession of a \acf{ssup} that
is non-kinetic in nature.  A distinction should be drawn, at this
point, between recommending that we provide arms to adversaries, and
examining the two scenarios.  This mitigation is more akin to asking
someone if they would prefer to drink muddy water or poison - neither
scenario is particularly appealing, but there is still a
clear\footnote{TeeHee! (xxx: remove this during the death-of-fun
editing phase)} winner.

If \acp{kasat} cannot be eliminated, are enormously destructive, are
poorly governed, and are likely to be used in conflict then the
orbital realm is in a very precarious position.  While the world
would, undoubtedly, be better off if \acp{kasat} did not exist, they
do exist.

Consider the following two scenarios:

\begin{enumerate}
\item Irresponsible actors like China and Russia have only \acp{kasat}
  to employ in the event of kinetic war, all but ensuring that they
  use them.

\item They have alternative weapons available to them which are less
  destructive.
\end{enumerate}

Option 2 is clearly less problematic.  While this conclusion is
unfortunate, it would likely result in reduced harm.

Unsavory though the conclusion may be, it is clear that western allies
and the world in general (including Russia/China) are better served by
ensuring that Russia/China possess alternatives (ideally impotent
alternatives).  These alternative systems could disrupt electrical
systems, limit maneuverability, occlude sensors and solar panels, or
damage radio systems; ultimately anything that eliminates the
strategic capability without generating the debris of a \ac{kasat}
would be a much better choice.  A satellite whose batteries have
overheated and no longer function, or whose solar panels have been
occluded, or whose electrical bus has been shorted is likely to be
just as useful to its operators as one which has been turned into a
puff of debris.  In combination with the next mitigation (Governance)
it might be possible to properly incentivize commanders to reach for
the less destructive alternatives.

There are other issues to consider, such as timing.  For example, if a
state of terrestrial kinetic war can be effectively staved off until
appropriate mitigations to their efficacy can be developed, then these
findings could become moot.  However, given the millennia-long
cat-and-mouse game developing kinetic weapons for terrestrial combat,
it seems exceedingly unlikely that all kinetic attacks will ever be
mitigated.  Even if it were assumed to be possible, the recent rise in
tensions between the western world and China/Russia makes it more
unlikely still. It could also be argued that having less catastrophic
options increases the likelihood that those weapons will be used.
After all, the deterrent effects of nuclear weapons are well
documented and generally effective\cite{getting-mad} but bullets are
fired with great regularity.

The question posed is not whether Russia or China should or should not
have ASAT missiles; they are too irresponsible to be entrusted with
them but they have them nonetheless.  The question is not whether
China will or will not invade Taiwan or whether Russia will or will
not invade Ukraine; China has made it quite clear that they intend
to\cite{china-cant-wait} and Russia already has. The question is not
whether or not China/Russia will or will not attack spacecraft; China
has already announced that they will\cite[p65]{dod-china-2020}, while
Russia has already destroyed a spacecraft as an apparent prelude to
war and has already conducted cyber attacks against American civilian
assets.\cite{viasat}\cite[02-17]{brian} The question posed is whether
the world is a better place with Xi and Putin being largely restricted
to the orbital version of nuclear weapons or if the world is perhaps a
bit less bleak with those rulers having the space version of
conventional arms at their disposal.

Perhaps the best summary of this thought is that, in some sense, it is
preferable to equip a child with a stick if they are already in
possession of a stick of dynamite.


\subsection{Mitigation of Premise 3: Management not Elimination}
Four nations are known to be capable of launching missiles to destroy
spacecraft.\cite{brian} If any of those four nations finds themselves
in an armed conflict with another major power, there will be
significant incentive to destroy opposing space assets; Chinese
defense scholars have notably stated that they would seek to ``blind
and deafen the enemy'' by ``destroying, damaging, and interfering with
the enemy's reconnaissance...and communications
satellites''\cite[p65]{dod-china-2020} In all four of these nations,
there is little or no evidence that proper \ac{cnc} safeguards exist
to ensure that the consequences of any strike order are properly
understood and that only those responsible parties can act.  Nuclear
\ac{cnc} safeguards, by contrast, are well-studied and known through
hard, if terrifyingly close, experience to at least be somewhat
effective.\cite{petrov-knows-best} There is even less evidence of
proper civilian oversight and policy to guide those decision-makers,
while simultaneously there is a great deal of evidence to suggest that
responsible actors are not in positions of appropriate influence in at
least two of those nations (Russia and China). Even worse, it is not
even clear that civilian oversight is even a meaningful concept in the
autocratic nations where this paper's findings most acutely apply:
Russia and China.\cite{dictator}

Despite all of the negatives associated with nuclear weapons, they are
still a major part of national arsenals.  They are effective weapons
in practice and they are effective deterrents\cite{getting-mad}, and
it seems unlikely that nations will ever willingly relinquish these
capabilities.  One critical difference between \acp{kasat} and nuclear
weapons, however, is that treaties governing the testing and use of
nuclear weapons exist in addition to national defense strategies etc.
Generally speaking, the world has some idea about how it would respond
to the use of nuclear weapons during war.\cite{old-timey-armageddon}

If \acp{kasat} are of great strategic value, and are as unlikely to be
eliminated as their terrestrial counterparts (nuclear weapons), then
perhaps attempts to ban them are in vain.  It is therefore proposed
that \ac{kasat} testing and usage be managed and mitigated by
international treaty, rather than eliminated entirely.

Assuming that the odds of banning \acp{kasat} is very close to zero,
pursuing that end is essentially equivalent to doing nothing and
accepting what may come.  Alternatively, management of these weapons
may yet be possible.  While highly imperfect, and perhaps better
regarded as a mess, nuclear weapons do receive some amount of
governance and oversight.  Using the history of successes and failures
in nuclear disarmament and pollution mitigation/remediation efforts,
it may be possible to enact treaties which constrain the testing and
use of \acp{kasat}.

Arguably, elimination of these weapons is impossible while management
of them is merely quite difficult.  If management stops even a single
high-altitude \ac{kasat} launch, then it can be said to have been
beneficial.
