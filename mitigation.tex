\section{Mitigation}

\subsection{Mitigation of Premise 1: Unclear}
While there is reason to believe that there are clandestine activities

designed to limit the efficacy of KKVs against individual assets,
there is little public evidence of these efforts, and even less about
wide-scale deployment.  Furthermore, the hypothesized methods below
would likely be cost prohibitive for commercial users.  As a thought
experiment, consider the following approaches:

\textbf{Stealth Satellites:} Decreasing RADAR cross-sections and IR
signatures might make targeting satellites more difficult and improve
survivability in the event of an attack.  See Appendix XXX for an
approximate SNR budget for the IR camera on the SM-3 interceptor from
Raytheon and the hypothesized necessary cooling method for a foreign
satellite to evade it.

\textbf{High Impulse Thrusters:} In principle, it might be feasible to
incorporate high-impulse thrusters to perform an evasive maneuver.
When coupled with high-accuracy Space Situational Awareness, it could
conceivably permit a spacecraft to avoid a collision.

\textbf{Kinetic Counteroffensive:} In Soviet Russia, Satellite Shoots
YOU!!!  In 196whatever, the Soviet Union launched and demonstrated a
machine gun on a spacecraft.  A relatively simple system such as Metal
Storm might be capable of initiating an early breakup of the KKV.
This approach has some obvious difficulties beyond basic execution
such as the generation of a large debris field still on track to
intersect with its target.

\textbf{ElectroMagnetic Pulse:} Utilizing newer wide bandgap
semiconductor systems, it might be possible to deliver an impulse with
sufficient field strength to electronically disable the KKV's
targeting and maneuvering systems.  Obvious difficulties include
things like targeting, and the ability to conduct the Eelectronic
Counter Measures (ECM) attack while there is still time to then evade
it, especially if the KKV is equipped with an inflatable system.  This
technology is currently being fielded by companies such as
Epirus.\cite[epirus]{fixme}

The obvious and straightforward methods one might conjure to avoid
collision with a modern KKV are, to put it charitably, problematic.
Lacking any obviously viable path to mitigate premise 1, and lacking
any public evidence of these efforts, it is assumed that Premise 1
will remain in effect for the foreseeable future.  If KKVs continue to
be effective, then there will be continued incentive to use them.

\subsection{Mitigation of Premise 2: Arming Adversaries}
This mitigation is, perhaps, the most disturbing of the three: the
world would be a better place if our adversaries had more weapons.  A
distinction should be drawn, at this point, between recommending that
we provide arms to adversaries, and examining the two scenarios.  This
mitigation is more akin to asking if you prefer to drink muddy water
or raw sewage - neither scenario is particularly appealing, but there
is still a clear winner.

If ASAT missiles cannot be eliminated, are enormously destructive, are
poorly governed, and are likely to be used in conflict then the
orbital realm is in a very precarious position.  While the world
would, undoubtedly, be better off if ASAT missiles did not exist, they
do exist, and are only one step removed from cyber-attacks with little
or no oversight in the hands of those who have already tested them in
an irresponsible manner.(cite analyses of fengyun and nudol) It is
believed that dictatorial leaders such as Xi, Putin, and Kim would
likely launch nuclear strikes rather than permit themselves to lose
power(cite that paper by that colonel and one other for funsies).  It
is hypothesized that military leadership might overrule a nuclear
strike order from one of these leaders (XXX: ask michael for
citation).  Would they override an ASAT strike order?

Consider, now, the two scenarios:

\begin{enumerate}
\item Irresponsible actors like China and Russia have only ASAT
  missiles to employ in the event of kinetic war, all but ensuring
  that they use them.

\item They have alternative weapons available to them which are less
  destructive.
\end{enumerate}

Option 2 is clearly less problematic.  While this conclusion is
unfortunate, it clearly results in less harm.

There are other issues to consider, such as timing.  For example, if a
state of terrestrial kinetic war can be effectively staved off until
appropriate mitigations to their efficacy can be developed, then these
findings become moot.  However, given the recent rise in tensions
between the western world and China/Russia, that scenario seems
unlikely.

As a more concrete example of this principle, consider the case of
Russia's invasion of Ukraine.  If they were armed only with
cyber-attacks and nuclear weapons then mushroom clouds would have
almost certainly sprouted.  As we saw, however, Russia had tanks and
rifles (and expendable riflemen) at their disposal and that is what
they chose to use.(citation even necessary?)  While still tragic,
Russia's invasion with tanks and rifles is less harmful than an
invasion with only nuclear weapons.  The same could be said of an
invasion of Taiwan by China.  In the ideal case, these nations would
be equipped with offensive technologies that they truly believe are of
utility, but in practice can be easily defeated.

The question posed is not whether Russia or China should or should not
have ASAT missiles; they are too irresponsible to be entrusted with
them but they have them nonetheless.  The question is not whether
China will or will not invade Taiwan or whether Russia will or will
not invade Ukraine; China has made it quite clear that they intend to
and Russia already has.(cite XXX) The question is not whether or not
China/Russia will or will not attack spacecraft; China has already
announced that they will, while Russia has already destroyed a
spacecraft as an apparent prelude to war and has already conducted
cyber attacks against American civilian assets.(cite nudol, viasat)
The question posed is whether the world is a better place with Xi and
Putin being largely restricted to the orbital version of nuclear
weapons or if the world is perhaps a bit less bleak with those rulers
having the space version of conventional arms at their disposal.

Unsavory though the conclusion may be, it is clear that western allies
and the world in general (including Russia/China) are better served by
ensuring that Russia/China possess alternatives (ideally impotent
alternatives).  These alternative systems could disrupt electrical
systems, limit maneuverability, occlude sensors and solar panels, or
damage radio systems; ultimately anything that eliminates the
strategic capability without generating the debris of a KKV would be a
much better choice.  A satellite whose batteries have overheated and
no longer function, or whose solar panels have been occluded, or whose
electrical bus has been shorted is likely to be just as useful to its
operators as one which has been turned into a puff of debris.  In
combination with the next mitigation (governance) it might be possible
to properly incentivize nations to reach for the less destructive
alternatives.

Perhaps the best summary of this thought is that, in some sense, it is
preferable to equip a child with a stick if they are already in
possession of a stick of dynamite.

The counterargument to this mitigation is that having less
catastrophic options increases the likelihood that they will be used.
The deterrent effects of nuclear weapons are well documented and
generally effective.(cite mad) However, bullets are fired with great
regularity.  A corollary is the decades-old debate around the use of
tactical nuclear weapons as a battlefield tool, rather than a
strategic deterrent.

\subsection{Mitigation of Premise 3: Management not Elimination}
Four nations are known to be capable of launching missiles to destroy
spacecraft, and three of them appear poised to form an alliance.(cite
nytimes article about india/china/russia) If any of those four nations
finds themselves in an armed conflict with another major power, there
will be significant incentive to destroy opposing space assets;
Chinese General XXX announced YYY in ZZZ.(dig up the citation from big
risks) In all four of these nations, there is little or no evidence
that proper CNC safeguards exist to ensure that the consequences of
any strike order are properly understood and that only those
responsible parties can act.  Nuclear CNC safeguards, by contrast, are
well-studied and known through hard, if terrifyingly close, experience
to at least be somewhat effective.(cite stuff) There is even less
evidence of proper civilian oversight and policy to guide those
decision-makers, while simultaneously there is a great deal of
evidence to suggest that responsible actors are not in positions of
appropriate influence in at least two of those nations (Russia and
China).  Even worse, it is not even clear that civilian oversight is a
meaningful concept in the autocratic nations where this paper's
findings most acutely apply: Russia and China.

Despite all of the negatives associated with nuclear weapons, they are
still a major part of national arsenals.  They are effective weapons
in practice (cite hiroshima) and they are effective deterrents (cite
XXX), and it seems unlikely that nations will ever willingly
relinquish these capabilities.  While many of the undesirable
consequences of nuclear weapons (such as nuclear fallout) possess
analogs to the effcts of KKVs (such as debris fields), it seems likely
that they will continue to be a part of national arsenals for the same
reasons that nuclear weapons will.  One critical difference between
ASAT missiles and nuclear weapons, however, is that treaties governing
the testing and use of nuclear weapons exist in addition to national
defense strategies etc.  Generally speaking, the world has some idea
about how it would respond to the use of nuclear weapons during war.
As President Biden recently announced, ``XXX''.

If ASAT weapons are of great strategic value, and are as unlikely to
be eliminated as their most closely related strategic weapons, then
perhaps attempts to ban them are in vain.  It is therefore proposed
that ASAT missile test and usage be managed and mitigated by
international treaty, rather than eliminated entirely.

If one assumes that the odds of banning these weapons is very close to
zero, then pursuing that end is essentially equivalent to doing
nothing and accepting what may come.  Alternatively, management of
these weapons may yet be possible.  While highly imperfect, and
perhaps better regarded as a mess, nuclear weapons do receive some
amount of governance and oversight.  Using the history of successes
and failures in nuclear disarmament efforts, it may be possible to
enact treaties which constrain the testing and use of ASAT missiles.

Arguably, elimination of these weapons is impossible while management
of them is merely quite difficult.  If management stops even a single
high-altitude ASAT missile launch, then it can be said to have been
beneficial.
