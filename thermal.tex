\section{Thermal Imaging Evasion}


The United States uses the \ac{sm3} for \ac{icbm} interception.  The
kinetic warhead employs a \ac{lwir} camera in the nose for target
identification.  While the efficacy has been questioned(cite the
nytimes article from the docs paper), decoy systems are frequently
actively cooled to fool the thermal imaging system(abm-me-not).  A
naive model was developed to approximate the necessary cooling to
result in a signal-to-noise ratio (SNR) of roughly 1 (meaning equal
amounts of signal and noise give-or-take a factor of 5).  Two primary
sources of thermal energy were considered 1) reflection of solar
emissions and 2) thermal emissions.  The mission concept assumes a
50cm x 50cm ``shield'' that can be oriented by the spacecraft to face
the incoming missile.  As shown below, the shield would need to have a
reflectivity of <= 1\% in the long-wave infrared band, and would need
to be cooled to <= 95 deg Kelvin (-178 C).  Hydrogen and Oxygen
propellants, as well as Argon, Nitrogen, and Helium are all suitable
for active cooling, while methane falls slightly short of the required
boiling temperature.

This model is naive, has not received engineering review, and may well
contain errors.  The point of comparison employed by a more complete
study of the topic done by the Union of Concerned Scientists discussed
a decoy at 77K, as opposed to the 95K discussed here.  Assumptions can
be found in figure \ref{figure::thermal::assumptions} and the resulting
budget in figure \ref{figure::thermal::budget}

\begin{figure}[ht!]
  \centering

  \begin{tblr}[
    ]{%
      colspec = {Q[c,m]|Q[c,m]|Q[c,m]},
      row{odd} = {gray!25}, row{even} = {white},
      row{1} = {gray!65},
    }
    {\bf Item}
    & {\bf Value}
    & {\bf Unit}
    \\

    KKV Lens Radius & 7 & $cm$ \\
    Reflectance & 0.01 & \\
    Fore-Optics Efficiency & 0.95 & \\
    Focusing-Optics Efficiency & 0.95 & \\
    Sensor Quantum Efficiency & 0.8 & \\
    Read Out Noise & 50 & $electrons$ \\
    Shield Area & 0.25 & $m^2$ \\
    Shield Temperature & 95 & $^{\circ}K$ \\
    Camera Frame Rate & 120 & $Hz$ \\
    KKV Range & 500 & $km$ \\

  \end{tblr}

  \caption{Initial assumptions in thermal imaging SNR model.}
  \label{figure::thermal::assumptions}
\end{figure}

\begin{figure}[ht!]
  \centering
  \begin{tblr}[
    ]{%
      colspec = {Q[c,m]|Q[c,m]|Q[c,m]},
      row{odd} = {gray!25}, row{even} = {white},
      row{1} = {gray!65},
    }
    {\bf Item}
    & {\bf Value}
    & {\bf Unit}
    \\

    Incident Solar Power & 1.98 & $dBW$ \\
    Reflected Solar Power & -18 & $dBW$ \\
    Thermal Radiation Emitted & -19.6 & $dBW$ \\
    Power Flux at Lens & -138 & $dB\frac{W}{m^2}$ \\
    Power at Lens & -156 & $dBW$ \\
    Shutter Time & 8.33 & $ms$ \\
    Signal Energy & -177 & $dBJ$ \\
    SNR & 1.85 & $dB$ \\
    SNR & 1.53:1 & \\

  \end{tblr}

  \caption{Resulting SNR budget.}
  \label{figure::thermal::budget}
\end{figure}

\subsection*{Consequential Factors}
\begin{itemize}

\item {\bf Distance between interceptor and target:} $R^2$ relationship, so
  2x decrease in distance results in 4x difference in SNR
  
\item {\bf Frame rate of the interceptor's camera:} Linear relationship, so
  a 2x decrease in framerate results in a 2x increase in SNR
  
\item {\bf Reflectance:} Linear relationship, so a 2x increase in
  reflectance of the shield results in a 2x increase in SNR

\item {\bf Lens radius:} $R^2$ relationship, so 2x increase in lens
  radius results in 4x difference in SNR

\item {\bf Shield Temperature:} Energy is radiated as a function of
  $T^4$ making the shield temperature highly impactful.  Liquid oxygen
  may get the job done, but it's probably better, as the Union of
  Concerned Scientists suggests, to utilize liquid nitrogen.

\end{itemize}
